\section{Background: Social creativity in CC}
\label{background}

Although we have adopted the term ``social creativity'' following
\cite{saunders12} and with the specific understanding developed above,
we could also refer in a similar spirit to situated, interactive,
communal, contextual, conversational, group, dialogical,
discourse-based, community-based, interaction-based, or
feedback-informed creativity.

The point is that creativity cannot exist in a vacuum.  The very
essence of creativity lives in its appreciation by the creative entity
themselves and its audience.  As we have remarked elsewhere,
creativity is in the eye of the beholder.  During the creative
process, self-evaluation abilities are crucial
\cite{poincare29,csik88}. Social creativity expands upon this
paradigm by bringing co-creators into the process, and creating works
that rely on dialogue, reflection, and multiple perspectives. The
``results'' may be steeped in process and will not always be based in
consensus.

The Four Ps of creativity -- the creative Person, Product, Process and
Press (i.e. environment) \cite{rhodes61,mackinnon70} -- have been
emphasised in general creativity research.  \emph{Pluralising} these
terms would call attention to a social dimension of creativity, and
leads to a more inclusive and encompassing approach to the study of
creativity -- one that accommodates multiple perspectives. The
Pluralised Ps remind us that it is not sufficient to model a lone
creator or to generate an attractive artefact.  To model creativity
more completely, we also need to consider the environment in which a
creative person operates, and how the environment is used in the
creative process.

Computational creativity research has achieved many successes in
computational generation of creative products. The question of how
these systems could adapt and learn from feedback to improve their
creativity, however, remains underexplored in computational creativity
despite evaluation being a pivotal contributory part of the creative
process. Researchers have generally preferred to take on the task of
generating artefacts that could be seen as creative, as a necessary
prior to the task of incorporating self-evaluation within a creative
system \cite{jordanous11iccc}.

Some notable exceptions exist, highlighting the importance of the
environment in which a creative system is situated \cite{mcgraw93,
  sosa09, perezyperez10MM, pease10, saunders12}, with some of this
work influenced by the DIFI (Domain-Individual-Field-Interaction)
framework \cite{csik88}. Generally, however, social interaction
between creative agents and their audience is an area which has been
neglected. Increased development of the interactivity of creative
systems, especially where this affects the way these systems works, is
pleasing to see and deserves further attention
\cite{coltonwiggins12}.

In the domain of poetry-generation, there have already been several attempts to simulate social creativity by incorporating multi-agent systems. 

In WASP \cite{gervas01,gervas10}  the social behavior is simulated by incorporating a \emph{cooperative society of readers/critics/editors/writers} consisting of specialized families of experts that cooperate during the poetry-generation process.

The McGONAGALL system \cite{manurung12} incorporates diverse modules as operators of evolutionary algorithms that produce poems fulfilling the constraints on \emph{grammar, meaning and poeticity}.  This approach allows pursuing several alternative solution paths in parallel, focusing on more promising results or coming back to former ideas. However this solution does not provide any communication between modules.

In MASTER system for computer-aided poetry generation \cite{kirke13} a \emph{society} of agents in various emotional states influences each other's moods with their pieces of poetry. The poetry-generation process is based on \emph{social learning} as the agents interact by reciting their own pieces of poetry to each other. The poetical form of the outputs is created by repeated words and sounds but the poems are not meaningful in the usual sense. 

\cite{misztal14} presents a Blackboard approach to poetry-generation in which independent specialized modules cooperate by sharing a common workspace -- the blackboard. This solution fulfills the assumptions of  the Global Workspace Theory of mind  \cite{baars97,baars03}. The experts exchange information with use of the global blackboard, however there is no direct communication among the modules and they do not receive any feedback about their artifacts.



\begin{mdframed}
\textbf{Other CC material?  Anything about poetry in particular?  It
  might be a good idea to mention Tristan Tzara, Brion Gysin, etc.
  Probably at least a few words about FloWr here as well.}

 E.g. \cite{jordanous10}?  Definitely we should cite
\cite{misztal2014poetry} and explain a bit about the MASTER system.
\end{mdframed}

%%% Local Variables: 
%%% mode: latex
%%% TeX-master: "poetryICCC"
%%% End: 
