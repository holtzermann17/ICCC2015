\section{Prime Ideals as a blend}
\label{sec:prime_ideals}

\subsection*{Introduction}
One of the most fundamental concepts of modern mathematics, which is
the basis of commutative algebra and a seminal ingredient of the
language of schemes in modern algebraic geometry is the concept of \emph{prime
ideal} \parencite{EGAI,eisenbud}. 
In this section, we will recover the concept of prime ideal of a
commutative ring with unity as a sort of partial or weakened blend
(i.e.~a blend for just some axioms of the input theories) between the
concepts of an \emph{ideal of a commutative ring with unity} (enriched with
the collection of all the ideals of the corresponding ring) and the
concept of a \emph{prime number} in the integers.

In order to obtain the desired space it is enough to consider
a more general version of the prime numbers (i.e.~a partial
version), namely, a monoid $(Z,*,1)$ with a ``special'' divisibility
relation $\dannydiv$.  In the blend, the generic space will just capture
the syntactic correspondences that we wish to identity in the
blend, since the blend will be basically the union of the
collection of axioms given on each space, doing the corresponding
identifications.

By doing just slight modifications to the input conceptual
spaces, we obtain also one of the most fundamental concepts of
algebraic number theory, the Dedekind domain \parencite[Theorem
37.1]{colemanmultidealtheory}, together with the collection of ideals
and prime ideals, as a blend of the concepts of noetherian domains
(with the set of ideals and prime ideals), and again a version of the
prime numbers in a very elementary form of the integers as before, but
adding the explicit axiom defining the upside-down divisibility
relation.  We can derive a new equivalent form of the definition of a
Dedekind domain based on the containment-division condition, which
suggests a new class of commutative rings, called here
containment-division rings (CDR).

We briefly present the conceptual spaces from the standard
mathematical point of view; due to limitations of space we 
present the corresponding
translation into the Common Algebraic Specification
Language (CASL) \cite{BidoitMosses2004} in an online annex.

~\\
~\\

\textbf{[JC: 1/2 to 3/4 of a page more to fill out this section, but
    without too many technical details.]}

\newpage
~\\
\newpage
% \input{prime-details}


%%% Local Variables: 
%%% mode: latex
%%% TeX-master: "mathsICCC"
%%% End: 

