\section{A Challenge Example for Blending} \label{galois}

\subsection{Computational Creativity via Blending}

The examples shown thus far in the paper have been examples of
blending in mathematics whose mechanisation has helped to identify
some novel and unexpected results. The blending itself was a one-stage
process where human input was required to identify the input
concepts. A more ambitious aim of the approach of applying blending to
the problem of computational creativity in mathematics, is to allow
search to be done over multiple blends and for the {\em process} of
blending to be controlled mechanically. In this section we describe
very informally a mathematical domain that seems in some ways a
natural candidate for a blending approach. A fuller mathematical
exposition would be required to support such an enterprise.


\subsection{Galois Theory}

%% This example is developed in a less formal manner than the previous.
%% It is included both to highlight some outstanding technical issues,
%% and to develop some broader theoretical points about the future
%% prospects and applications of our approach.  

In its most straightforward formulation, Galois theory develops a
relationship between a polynomial $p(x)$ with coefficients in some
field $F$, the extension of $K$ of $F$ (written ``$K/F$'') containing
all of the roots of $p(x)$ in the algebraic closure of $F$, and the
group $\mathbf{Gal}(K)$ of automorphisms of $K/F$ that fix the
elements of $F$.  The fundamental theorem of Galois theory states that
there is a bijection between the subfields of $K/F$ and the subgroups
of $\mathbf{Gal}(K)$; namely, subgroups correspond to their fixed
fields.  Using this correspondence, properties of polynomials can be
derived, most famously the fact that quintic polynomials cannot be
solved by algebraic operations and the extraction of roots.

We do not propose to reconstruct much of the theory here, but note
that already in this basic account there are several steps that seem
compellingly ``blend-like.''

In the first place, that would describes the notion of a field
extension quite well.  $E$ is an extension of $F$ if $F$ is a subfield of $E$.
We could derive the extension relationship from the input concepts $E$
and $F$ by ``taking everything additional from $E$ and adding it to
$F$.''  This is made specific in the process of \emph{adjoining}
elements to a field, which simply means to augment the field with all
formal finite sums and products of the adjoined elements with
coefficients in the base field.

Second, the notion of the \emph{splitting field} of a polynomial,
namely the special extension $K/F$ containing all of the roots of
$p(x)$.  This could be formed conceptually by combining the concept
``\emph{the roots of a polynomial $p(x)$ with coefficients in a field
  $F$}'' and the concept ``\emph{a field extension $E/F$ formed by
  adjoining certain elements to $F$}.''  Formally, the roots of $p(x)$
are not part of in the second concept, and they must be put in
correspondence with the ``certain elements.''  Note that formulating
the concept of a splitting field in this way is different from proving
that a splitting field always exists.  It does, however, and the proof
(by induction on the degree of the polynomial) works by successively
adjoining elements to $F$.  This gives an inkling of the idea that
blending could be used as a proof step.

As above, we could then form the concept of $\mathbf{Gal}(K)$ by blending
at the conceptual level.  This time, there would be several constituent pieces:
``\emph{the roots of a polynomial $p(x)$ with coefficients in a field $F$},''
``\emph{the splitting field of $p(x)$},''
``\emph{the group of automorphisms of a field extension $E$},''
``\emph{the automorphisms that fix $F$}.''

Finally, assuming that we have built $\mathbf{Gal}(K)$ in this
fashion, we would like to know some of its properties.  Consider the
claim that \emph{elements of $\mathbf{Gal}(K)$ permute the roots of
  $f$}.  This time, instead of being purely conceptual, we want to
work at the \emph{process} level, and consider before-and-after
descriptions of the result of applying $\varphi\in\mathbf{Gal}(K)$ to
some $r$ with the property $p(r)=0$.  This is similar in some ways to
the ``Riddle of the Buddhist Monk'' \cite{Fau98} which is cited as
an example of the power of blending.\footnote{Fauconnier and Turner
  cite Arthur Koestler: ``A Buddhist monk begins at dawn one day
  walking up a mountain, reaches the top at sunset, meditates at the
  top for several days until one dawn when he begins to walk back to
  the foot of the mountain, which he reaches at sunset.  Making no
  assumptions about his starting or stopping or about his pace during
  the trips, prove that there is a place on the path which he occupies
  at the same hour of the day on the two separate journeys.''}
However, this time the generic space is not a simple geometric machine,
but rather an algebraic machine with several moving parts.

The proof of the claim is as follows.  If $p(r)=0$, then $\varphi
p(r)=\varphi 0$.  Since $\varphi$ is an automorphism, $\varphi 0 = 0$;
and furthermore $\varphi$ distributes over the sums and products that
make up the polynomial $p(x)$ and fixes its coefficients, therefore
$\varphi p(r)=p(\varphi r)$.  Chaining the equalities together, we
have $p(\varphi r)=0$.

In short, the proof is a fairly direct result of combining the
definitions.  \textcite{Goguen92sheafsemantics} suggests that
``combination is colimit.''  Can we realise the proof through (one or
several) colimit operations?  And is there anything special about this
proof?  Apart from these more theoretical questions, the foregoing
discussion raises the following technical issues:

\begin{description}
\item[Field Extension] When reasoning about polynomials, it is useful
  to distinguish the three separate types -- those of $E$, those of
  $F$ and those of $E/F$ as a supertype. Using blending machinery
  removes the distinction between these types.
\item[Splitting Field Extension Theorem] A challenging but creative
  step is to discover the theorem that extending $F$ {\em only} with
  the roots of $f(x)$ forms a field.
\item[Automorphisms] Currently there is no way of computing colimits
  with higher order symbols mechanically, so we are constrained to do
  this on paper.
\end{description}

% \subsection{Issues raised}

We have investigated the possibility of formalising Galois theory, and
note here some of the possibilties for blend computations, along with
some issues we need to address.

\subsubsection{Field Extension as a Blend}

The first possibility here is that one could exploit blending
machinery in order to discover the notion of a field extension. Take a
field $F$ and adjoining it with an element $a$ to form $F(a)$ is a
form of a blend of the conceptual space containing the field $F$ and
the conceptual space containing element $a$. Equally for a field
extension one could envisage blending a conceptual space containing
field $F$ with a conceptual space containing a set of elements
$E$. The field axioms then hold for $F$ and $E/F$, but not for the
elements of $E$.

Although this seems a natural blend, in reality when reasoning about
polynomials with rational coefficients, it is useful to distinguish
the three separate types -- those of $E$, those of $F$ and those of
$E/F$ as a supertype. Using blending machinery removes the distinction
between these types.

\subsubsection{Splitting Field Extension Theorem}

We would like to be able to use blending to discover that there exists a
field $F$ extended with roots of polynomial $f(x)$, where the coefficients of $f(x)$ are in $F$. The more challenging but creative step is to discover that extending $F$ {\em only} with the roots of $f(x)$ forms a field. We have investigated the possibility of blending the concept of the root of a polynomial with rational coefficients with a field of rationals and then trying to prove the theorem from the resulting axioms. Discovery of the theorem itself is the real creative step and is an example of running the blend. This is ambitious and is further work.

\subsubsection{Automorphisms}

In order to discover the most interesting and surprising Galois
theorem, that automorphisms fixing the field $F$ permute the elements
of $E$ in field extension $E/F$, we need to be able formalise the
notion of an automorphism. The most natural way to do this is in the
machinery that we currently use is to define an automorphism $\alpha$
as a higher order function: $\alpha:\; E/F \to E/F$. Currently there
is no way of computing colimits with higher order symbols
mechanically, so we are constrained to do this on paper.
%% There are various technical questions \textbf{(@Alan, @Ewen, @Felix)},
%% but from a naive mathematical standpoint the first issue is: is it
%% always clear how to combine the relevant facts?  And a related
%% question: is it always clear what the relevant facts actually are?

%% If blending is the realisation of ``combinatorial creativity'' why are
%% we not swamped by the combinatorial explosion of possible things to
%% combine?


%%% Local Variables: 
%%% mode: latex
%%% TeX-master: "mathsICCC"
%%% End: 
