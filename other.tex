\section{Related Example} \label{galois}

\subsection{Galois Theory}
% Danny, Joe, Felix, Ewen

This example is developed in a less formal manner than the previous.
It is included both to highlight some outstanding technical issues,
and to develop some broader theoretical points about the future
prospects and applications of our approach.  

In its most straightforward formulation, Galois theory develops a
relationship between a polynomial $f(x)$ with coefficients in some
field $F$, the extension of $K$ of $F$ (written ``$K/F$'') containing
all of the roots of $f(x)$, and the group $\mathbf{Gal}(K)$ of
automorphisms of $K/F$ that fix the elements of $F$.  The fundamental
theorem of Galois theory states that there is a bijection between the
subfields of $K/F$ and the subgroups of $\mathbf{Gal}(K)$;
namely, subgroups correspond to their fixed fields.  Using this
correspondence, properties of polynomials can be derived, most
famously the fact that quintic polynomials cannot be solved by
algebraic operations and the extraction of roots.  

We do not propose to reconstruct much of the theory here, but note
that already in this basic account there are several steps that seem
compellingly ``blend-like.''

In the first place, that would describes the notion of a field
extension quite well.  $E$ is an extension of $F$ if $E \supseteq F$.
We could derive the extension relationship from the input concepts $E$
and $F$ by ``taking everything additional from $E$ and adding it to
$F$.''  This is made specific in the process of \emph{adjoining}
elements to a field, which simply means to augment the field with all
formal finite sums and products of the adjoined elements with
coefficients in the base field.

Second, the notion of the \emph{splitting field} of a polynomial,
namely the special extension $K/F$ containing all of the roots of
$f(x)$.  This could be formed conceptually by combining the concept
``\emph{the roots of a polynomial $f$ with coefficients in a field
  $F$}'' and the concept ``\emph{a field extension $E/F$ formed by
  adjoining certain elements to $F$}.''  Formally, the roots of $f$
are not part of in the second concept, and they must be put in
correspondence with the ``certain elements.''  Note that formulating
the concept of a splitting field in this way is different from proving
that a splitting field always exists.  It does, however, and the proof
(by induction on the degree of the polynomial) works by successively
adjoining elements to $F$.  This gives an inkling of the idea that
blending could be used as a proof step.

As above, we could then form the concept of $\mathbf{Gal}(K)$ by blending
at the conceptual level.  This time, there would be several constituent pieces:
``\emph{the roots of a polynomial $f$ with coefficients in a field $F$},''
``\emph{the splitting field of $f$},''
``\emph{the group of automorphisms of a field extension $E$},''
``\emph{the automorphisms that fix $F$}.''

Finally, assuming that we have built $\mathbf{Gal}(K)$ in this
fashion, we would like to know some of its properties.  Consider the
claim that \emph{elements of $\mathbf{Gal}(K)$ permute the roots of
  $f$}.  This time, instead of being purely conceptual, we want to
work at the \emph{process} level, and consider before-and-after
descriptions of the result of applying $\varphi\in\mathbf{Gal}(K)$ to
some $r$ with the property $f(r)=0$.  This is similar in some ways to
the ``Riddle of the Buddhist Monk'' \cite{FaTu98b} which is cited as
an example of the power of blending.\footnote{Fauconnier and Turner
  cite Arthur Koestler: ``A Buddhist monk begins at dawn one day
  walking up a mountain, reaches the top at sunset, meditates at the
  top for several days until one dawn when he begins to walk back to
  the foot of the mountain, which he reaches at sunset.  Making no
  assumptions about his starting or stopping or about his pace during
  the trips, prove that there is a place on the path which he occupies
  at the same hour of the day on the two separate journeys.''}
However, this time the generic space is not a simple geometric machine,
but rather an algebraic machine with several moving parts.

The proof of the claim is as follows.  If $f(r)=0$, then $\varphi
f(r)=\varphi 0$.  Since $\varphi$ is an automorphism, $\varphi 0 = 0$;
and furthermore $\varphi$ distributes over the sums and products that
make up the polynomial $f$ and fixes its coefficients, therefore
$\varphi f(r)=f(\varphi r)$.  Chaining the equalities together, we
have $f(\varphi r)=0$.

In short, the proof is a fairly direct result of combining ``what it
means to be a root,'' ``what it means to be an automorphism,''
``what it means to say that the automorphism fixes elements of $F$,''
and
``what it means to be a polynomial with coefficients in the
field $F$.''
Indeed, the proof is in some sense the only thing it could be if one
knows the definitions.  

\cite{Go99c} suggests that ``combination is colimit.''  Can we realise
the proof through (one or several) colimit operations?  I.e.~is the
proof a blend?  And is there anything special about this proof?

\subsection{Issues raised}

There are various technical questions \textbf{(@Alan, @Ewen, @Felix)},
but from a naive mathematical standpoint the first issue is: is it
always clear how to combine the relevant facts?  And a related
question: is it always clear what the relevant facts actually are?

% Alan, Ewen, Felix

%%% Local Variables: 
%%% mode: latex
%%% TeX-master: "mathsICCC"
%%% End: 
