\documentclass[letterpaper]{article}
\usepackage{iccc}
\usepackage{times}
\usepackage{helvet}
\usepackage{courier}
\usepackage[firstpage]{draftwatermark}
\SetWatermarkText{\shortstack{Draft\\\today\currenttime}}
\SetWatermarkScale{0.75}
\usepackage[T1]{fontenc}
\usepackage{csquotes}
\usepackage[utf8]{inputenc}
\usepackage[british]{babel}
\usepackage[style=apa,natbib,backend=biber]{biblatex}
\usepackage[ddmmyyyy]{datetime}
\renewcommand*{\multicitedelim}{\addsemicolon\space}%
\usepackage{xcolor}
\usepackage[
  bookmarks=false,
  pdfpagelabels=false,
  hyperfootnotes=false,
  hyperindex=false,
  pageanchor=false,
%  colorlinks,
  citebordercolor=white
]{hyperref}

\DeclareLanguageMapping{british}{british-apa}
% \addbibresource{papers.bib}
\addbibresource{mathsICCC.bib} %
%% \urlstyle{tt}
\AtBeginBibliography{\def\UrlFont{\footnotesize\tt}}
\renewcommand*{\finalnamedelim}{%
  \ifnumgreater{\value{liststop}}{2}{\finalandcomma}{}%
  \addspace\&\&\space}
\AtEveryCitekey{\renewcommand*{\finalnamedelim}{%
  \ifthenelse{\value{listcount}>\maxprtauth}
    {}
    {\ifthenelse{\value{liststop}>2}
       {\finalandcomma\addspace\bibstring{and}\space}
       {\addspace\bibstring{and}\space}}}}

\AtBeginDocument{\renewcommand\finalandcomma{\addcomma}}
\AtBeginBibliography{%
  \renewcommand*{\finalnamedelim}{%
    \ifthenelse{\value{listcount}>\maxprtauth}
      {}
      {\finalandcomma\addspace\bibstring{and}\space}}}

% \pdfinfo{
% /Title (Test)
% /Subject (Proceedings of ICCC)
% /Author (ICCC)}
% The file iccc.sty is the style file for ICCC proceedings.

\title{Chasing the blend}
\author{Several\\
check if this should be blind submission?
}
\setcounter{secnumdepth}{0}

\begin{document} 
\maketitle
\begin{abstract}
\begin{quote}
We model the mathematical process whereby new mathematical
theories are produced, involving shared and individual creativity.
Here we provide rational reconstructions of some developments
from mathematical history;  our longer-term goal is to support
machine and human mathematical creativity.
\end{quote}
\end{abstract}

\section{Introduction}
\label{sec:intro}

% \begin{itemize}
% \item Why poetry?
% \item Why FloWr?
% \item Why Workshop?
% \end{itemize}

We are writing in a large part to champion Alan Turing's
proposal that intelligent machines should ``be able to converse with
each other to sharpen their wits'' \cite{turing-intelligent}. 
%
The formalism that we propose builds on the notion of social cybernetics that flows
from the following propositions of Heinz von Foerster's, which he uses to theorise systems
in which participants can responsibly specify their own roles in relationship to
other system participants:

\begin{quote}
``Anything said is said \emph{by} an observer.'' \\
``Anything said is said \emph{to} an observer.''\\
\sourceatright{\cite{von2003cybernetics}}
\end{quote}

According to Jaako Seikkula and Tom Arnkil, who draw on the philosophical and literary analysis of Mikhail Bakhtin \cite{bakhtin2010toward,bakhtin1984problems} in their approach to psychosocial work,
\begin{quote}
``Dialogues could be called `the art of crossing boundaries'.  Instead of trying to control others, the parties reach out towards each other to hear their views better, to generate shared languages and to join resources.''
\sourceatright{\cite[p. 23]{seikkula2014open}}
\end{quote}

This paper outlines a study of social creativity with a dialogical emphasis, taking computer
poetry as our working domain.  It uses the Writer's Workshop model
\cite{gabriel2002writer} as the virtual laboratory in which to conduct a thought experiment.
The findings of our study are applied to the FloWr system \cite{charnley2014flowr}. 
We focus on the following questions in turn:
%\subsection{Outline}


\begin{itemize}[label=--,itemsep=0pt]
% BACKGROUND
% - a few more details
\item How has the social dimension of creativity been explored in CC to date? 
% PHILOSOPHY AND METHODS
% - break into subsections and add tools, integrate to table
%CG: Modified this to connect rosies methodology to the workshop and CC
\item How can a created artefact tell us about its making, and what can this contribute to CC?
% DESIGN
\item How can computer poetry contribute to developing a process-based theory of poetics? 
% FLOWR
% - Teresa to revise 
\item What would have to change about the FloWr system to implement the computational poetry workshop approach?
% DISCUSSION
% - keep it?
\item What are the pros and cons of the workshop approach?
% CONCLUSIONS
% - taking into account Christian's comment, I've revised this and will edit the conclusion to align.  JC.
\item What might be the future role of dialogue in CC?
\end{itemize}

%%% Not nec. needed
%%% In the background section, we will focus on social creativity in the computational creativity domain and more specifically in computer poetry. This will be followed by a Methods section, in which we ...

%%% Local Variables: 
%%% mode: latex
%%% TeX-master: "poetryICCC"
%%% End: 

\section{Background}
\label{background}

\subsection{Social creativity in CC}

Although we have adopted the term ``social creativity'' following
\cite{saunders12} and with the specific understanding developed above,
we could also refer in a similar spirit to situated, interactive,
communal, contextual, conversational, group, dialogical,
discourse-based, community-based, interaction-based, or
feedback-informed creativity.

The point is that creativity cannot exist in a vacuum.  The very
essence of creativity lives in its appreciation by the creative entity
themselves and its audience.  As we have remarked elsewhere,
creativity is in the eye of the beholder.  During the creative
process, self-evaluation abilities are crucial
\cite{poincare29,csik88}. Social creativity expands upon this
paradigm by bringing co-creators into the process, and creating works
that rely on dialogue, reflection, and multiple perspectives. The
``results'' may be steeped in process and will not always be based in
consensus.

The Four Ps of creativity -- the creative Person, Product, Process and
Press (i.e. environment) \cite{rhodes61,mackinnon70} -- have been
emphasised in general creativity research.  \emph{Pluralising} these
terms would call attention to a social dimension of creativity, and
leads to a more inclusive and encompassing approach to the study of
creativity -- one that accommodates multiple perspectives. The
Pluralised Ps remind us that it is not sufficient to model a lone
creator or to generate an attractive artefact.  To model creativity
more completely, we also need to consider the environment in which a
creative person operates, and how the environment is used in the
creative process.

Computational creativity research has achieved many successes in
computational generation of creative products. The question of how
these systems could adapt and learn from feedback to improve their
creativity, however, remains underexplored in computational creativity
despite evaluation being a pivotal contributory part of the creative
process. Researchers have generally preferred to take on the task of
generating artefacts that could be seen as creative, as a necessary
prior to the task of incorporating self-evaluation within a creative
system \cite{jordanous11iccc}.

Some notable exceptions exist, highlighting the importance of the
environment in which a creative system is situated \cite{mcgraw93,
  sosa09, perezyperez10MM, pease10, saunders12}, with some of this
work influenced by the DIFI (Domain-Individual-Field-Interaction)
framework \cite{csik88}. Generally, however, social interaction
between creative agents and their audience is an area which has been
neglected. Increased development of the interactivity of creative
systems, especially where this affects the way these systems works, is
pleasing to see and deserves further attention
\cite{coltonwiggins12}.

\begin{mdframed}
\textbf{Other CC material?  Anything about poetry in particular?  It
  might be a good idea to mention Tristan Tzara, Brion Gysin, etc.
  Probably at least a few words about FloWr here as well.}

 E.g. \cite{jordanous10}?  Definitely we should cite
\cite{misztal2014poetry} and explain a bit about the MASTER system.
\end{mdframed}

\subsection{Writer's Workshops}

Quoting \cite[pp. 2--3]{gabriel2002writer}:

\begin{quote}
The original idea behind the writers' workshop was to do a \emph{close
  reading} of a work, to use the term F. R. Leavis coined for the
practice of looking at the words on the page rather than the
intentions of the author or the historical and aesthetic context of
the work.  Under this philosophy, the workshop doesn't care much what
the author feels about what he or she wrote, only what's on the page.
This corresponds to the philosophy of the New Critics, which held that
the work was its own ``being,'' with its own internal consistency and
coherence, which could be studied apart from the author.  Moreover,
this approach is nearly identical to that of the Russian formalists,
who thought that the proper approach to literature was to study how
literary texts actually worked, their structures and devices.
\end{quote}

Framing and any other contextualisation of the work \emph{as it is
  intended to be presented} is permitted, and receives critical attention.

In \cite{serendipity-arxiv}, we described a template for a pattern
language for interactions in a computational poetry workshop, closely
following Gabriel's outline of the relevant steps: {\tt presentation},
{\tt listening}, {\tt feedback}, {\tt questions}, and {\tt
  reflections}.\footnote{To this should be added the potential for
  real-time {\tt replies} by critics to {\tt questions} asked by the
  presenting author, before subsequent ``offline'' {\tt reflections}.}
We used this template to expand several of the patterns of serendipity
described by \cite{van1994anatomy}, showing how they could be used to
foster discovery and invention in a workshop environment.



%%% Local Variables: 
%%% mode: latex
%%% TeX-master: "poetryICCC"
%%% End: 

\input{infinity}
\input{prime}
\input{other}
\input{evaluation}
\section{Conclusions}
\label{sec:conc}

We have made the case for dialogicity in CC and in CC research,
building on a comparison of different ways to read poems, and
considering the potential of an existing software system to support
dialogue at various levels.

Our requirements for communication points to the need for program
elements to be able to learn and adapt.  Genetic programming
\cite{koza1992genetic} and related methods offer a certain precedent
for this way of working.

The current paper has focused on more global concerns.  The current
FloWr ``ecosystem'' focuses on features like user interface ease for
human programmers, plus reusability of modules.  The next steps should
involve an agent-based redesign.  We have sketched related work in
Section \ref{sec:background}.  Off-the-shelf open source software
exists to support some relevant basic features.

However, the question of \emph{context} has not been adequately
addressed either in CC or in mainstream programming.  While we do not
agree with interpretations of artwork that macro-reductionistic in the
sense that ``the environment determines the artwork'' nevertheless
there are important contextual effects \cite{geertz1976art}, and
artwork generally involves an engagement with context.
Philosophically this situates the work in the realms of
\emph{pragmatics} \cite{sep-pragmatics} and \emph{metalinguistics}
\cite{gombert1994development}.  Practically speaking, the questions
relate to building sophistication in the specific ``metalanguage'' of
programming.  The programming tasks involved are not necessarily ends
in themselves, however, but oriented, in the first place towards
building better artefacts.  These too have a goal, which (at least in
a simple case) is to communicate.

The graphical programming environment that shipped with the 1984 robot
simulation game ChipWits, not taken seriously as a competitor to
FloWr, does nevertheless have some useful programming facilities, like
the ability to loop and run different paths in the flowchart depending
on conditions, and the ability for one flowchart to reference another
one, as with spreadsheets.  However, it does not have the ability to
self-program, which is the essence of the proposal for FloWr.  The
ability to generate-and-check (using a population-based mechanism) and
more importantly, the ability to learn over time (as we explored from
a formal perspective in \cite{colton-assessingprogress}) are two other
features that should come standard in future versions of FloWr.  We
want to be able to take account of the abundance of available
information, to introduce ``noise'', and fully take account of the
existing ``framing'' in relationship to the context/situation.

Although social creativity has been explored in CC, it tends to be an
exception rather than the rule.  In future work, we'd like to be able
to say ``We've done it!'' although to develop a concrete proof of
concept we will have to focus on a few simple metrics (like
musicality).

There is also a ``social engineering'' component to this paper, hoping
to motivate a new approach to research evaluation in CC.  Between the
kind of design research carried out in this paper and a large-scale
double-blind study that would ``prove'' the superiority of a social
approach is something more contextual, involving the actual practices
and abilities of participants
\cite[pp. 167--185]{seikkula2006dialogical}.  We sketched an approach
to the evaluation of poems, but similar thinking can help develop an
evaluation of programs.  The question of how much architecture should
be shared in the CC community: do we need a shared ``CCyc'' platform,
or should we ``let a hundred flowers bloom''?  One moderate approach
would be to revive the \emph{floral games} of the troubadours.

%%% Local Variables: 
%%% mode: latex
%%% TeX-master: "mathsICCC"
%%% End: 


\nocite{Weil60}
\nocite{Boden90}
\nocite{nunez05}
\nocite{Lak00}
\nocite{Gog05,Gog99}

\printbibliography

\end{document}
