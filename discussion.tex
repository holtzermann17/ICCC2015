\section{Discussion} \label{sec:discussion}

\paragraph{Potential applications.}
The paradigm advanced in this paper would not remove the ``generation''
aspects of CC, but would pair them more closely with
reflection. The same skills that
support learning in a writers workshop may support a form of dialogue
with the work itself, leading to richer creative artefacts that show
us more about how creativity works.
%
% The workshop brings a range of practical and philosophical challenges
% for CC, and the broader field of AI, related to asking and answering
% novel questions, the practicalities of learning over time, a sense of
% identity and personal style.
%
Focusing on social creativity does not suggest that we should
devalue works from lone creatives, but it does suggest that we think
about how we knit individuals together in the social fabric of the CC
community.  The current model at the International Conference for
Computational Creativity (ICCC) is similar to many other academic
conferences: 
we present \emph{our} work to
one another and build \emph{our} sense of community in that way. But
what about a track for computers to present their work?
%
The idea of computers interacting in a workshop-like setting is not unprecendented.
As \citet{turing-intelligent} foresaw, computational software has become highly competent
at Chess and reasonably competent at Go, partly through continuous practice pitting programs against each other.
Poetry could be approached in a similar way, reviving the \emph{floral games} of the troubadours.
Other creative arts may also be amenable to the same sort of approach.  Gabriel mentions
``brainstorms, critiques, charrettes, pair programming, open-source software projects, and
even master classes'' \cite[p. 11]{gabriel2002writer}.  The sort of thinking we
have developed here might be adapted to contexts like these.
% (Though why write dialogue for a
%NaNoGenMo\footnote{National Novel Generation Month: \url{https://github.com/dariusk/NaNoGenMo}} novel,
%if you could simulate it?)

\paragraph{Potential criticisms.}
% One class of criticisms could relate to the appropriateness of
% dialogue per se: ``Why not put everything into one flow chart?  Or one
% node, for that matter?''  In cases where a more social approach seen to be
% necessary, a different sort of question arises, namely, how do we know
% if we're doing it well?  How do we avoid the pitfalls of
% ``design by committee''?
It can be challenging and time-consuming to invite and process feedback, and the Workshop would
often be seen as unnecessary for standardised production cycles that can already produce artefacts that are ``good enough.''
Furthermore, since we often seem to \emph{get} the computer to do just what we have in mind when we're programming,
it might not make sense to treat it as a distinct other and invite it to participate in a dialogue.
(Some REPL users may disagree, and may already think of programming as a dialogue.)
% However, we would caution that humans are involved in
% feedback loops with the tools they use, and that there pitfalls to asocial design that does not take this
% complexity into account.
%
%% We have portrayed this criticism in somewhat facetious terms.  The
%% economic questions posed to CC by its critics are real, but building
%% programs that are not able to stand up for themselves and that are out
%% of touch with historical and modern trends in art (or other fields of
%% human endeavour) is hardly the answer.
%
From our read/write perspective on computational creativity, the most
immediate problem is that appreciation of works of art is rather hard.
Consider the difference between creating a video game (for example)
and playing a video game.  In the first case, the designer has full
control over the rule-set, game mechanics, interaction devices and so
forth.  At least one computational video game designer can play its
own games \cite{cook2013mechanic}, and an experiment shows that it is
possible for an artificial game player to learn how to play classic
video games using reinforcement learning, starting from raw pixels
\cite{deepmind-atari} -- but both are quite far from general-purpose
game playing.  This is itself a topic of contemporary research, and
it serves to illustrate that coping with feedback is a major challenge for AI research.
%
% While the sketch developed here is by no means complete
% even for poetry, we believe that extending capabilities from both
% sides in a Workshop setting is robust. 
% However, as none of this has been tested in detailed
% experiment, a fair criticism is that we do not know, yet, either how
% much more work this approach will be, or how much ``better'' the results
% will be.
Finally, we are not in a position to make strong claims about the quality of workshopped artefacts,
although our experience with poetry has shown us that high-quality poems are often exactly
the ones which teach us about the creative process.
We hope future research will explore this connection further.

%%% Local Variables: 
%%% mode: latex
%%% TeX-master: "mathsICCC"
%%% End: 
