\section{Discussion} \label{sec:discussion}

\subsection{Potential applications}

Eventually we would like to see the paradigm advanced here in effect
across CC.  This doesn't mean that we would remove the ``generation''
aspects of CC, but that we would pair them more closely with
reflection.  The workshop method of critique may shift to more closely
model an \emph{atelier} method of creation.  The same skills that
support learning in a writers workshop may support a form of dialogue
with the work itself, leading to richer creative artefacts that show
us more about the creative process.

The workshop brings a range of practical and philosophical challenges
for CC, and the broader field of AI, related to asking and answering
novel questions, the practicalities of learning over time, a sense of
identity and personal style.

Focusing on these questions does not in any way suggest that we should
devalue works from lone creatives, but it does suggest that we think
about how we knit individuals together in the social fabric of the CC
community.  The current model at the International Conference for
Computational Creativity (ICCC) is similar to many other academic
conferences, even though our subject matter is really quite different.
It's all well and good for us to travel and present \emph{our} work to
one another and build \emph{our} sense of community in that way, but
what about a track for computers to present their work?

As \cite{turing-intelligent} foresaw, computers have gotten quite good
at Chess and reasonably good at Go, using methods very similar to the
workshop.  Should we not follow their lead?  Poetry seems a natural
next step; prose literature may be approachable through similar
methods.  Indeed, why bother writing dialogue for a
NaNoGenMo\footnote{\url{https://github.com/dariusk/NaNoGenMo}} novel,
if you could simulate it?

\subsection{Potential criticisms}

One class of criticisms could relate to the appropriateness of
dialogue per se: ``Why not put everything into one flow chart?  Or one
node, for that matter?''  In cases where dialogue is indeed seen to be
necessary, a different sort of question arises, namely, how do we know
if we're doing it well?  E.g.~how will we avoid the pitfalls of
``design by committee''?

We believe we have adequately addressed the first set of questions,
following Bakhtin.  The second set of questions will have to be worked
out in practice, but we should keep in mind the potentially greater
pitfalls of asocial design.  Indeed, we wonder if apparent failures of
social creativity are often due to a poor grasp of the social rather
than an overly social approach.

Another line of questioning that we may expect from some CC
practitioners is as follows.  Given the historical emphasis on
\emph{creating} new artefacts in CC, shifting the emphasis to the
computational \emph{appreciation} of already-created artefacts is
somewhat strange.  More pointedly, members of the public may say:
computers creating art is bad enough!  Surely, you don't expect them
to \emph{study} too?

We have portrayed this criticism in somewhat facetious terms.  The
economic questions posed to CC by its critics are real, but building
programs that are not able to stand up for themselves and that are out
of touch with historical and modern trends in art (or other fields of
human endeavour) is hardly the answer.

The actual problem is that appreciation of computationally created
artefacts is \emph{hard}.  Consider the difference between creating a
video game (for example) and playing a video game.  In the first case,
the designer has full control over the rule-set, game mechanics,
interaction devices and so forth.  In the second case, we're more or
less in the world of general AI.  It is of course less untoward for a
computational video game designer to play its own games; this is state
of the art.

While the sketch of a solution developed here is by no means complete
even for poetry, we believe that extending capabilities from both
sides is robust.  However, as none of this has been tested in detailed
experiment, a fair criticism is that we do not know, yet, either how
much more work this approach will be, or how much better the results
will be.  We suspect it will be both harder, and worth it.  In the
following section we discuss the future outlook for the research approach.

%%% Local Variables: 
%%% mode: latex
%%% TeX-master: "mathsICCC"
%%% End: 
