\section{Conclusions}
\label{sec:conc}

The examples presented in this paper trace the development of the
blending approach.  The early examples illustrate the reconstruction
of certain mathematical objects.  The middle examples begin to ``kick
back'' -- we recover consistent blends but only in the PID case.  We
noticed also that ``partial combinations'' are a better reflection of
mathematical practice than ``total combinations.''  Finally, the
future-oriented examples present several challenges to implementation,
but also suggest that blending, when used at its full power in a
process-oriented setting, could offer a novel approach, relevant not
only in computer mathematics, but in learning science.

As regards computational creativity: in the current paper we have
focused on reconstructions, but analysis offered here shows how this
work is a building block that will be useful for future developments
that are able to reason more flexibly about mathematical problems --
and potentially find and define new ones.

%%% Local Variables: 
%%% mode: latex
%%% TeX-master: "mathsICCC"
%%% End: 
