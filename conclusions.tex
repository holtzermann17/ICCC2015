\section{Conclusions}
\label{sec:conc}

The examples presented in this paper trace the development of the
blending approach.  The early examples illustrate the reconstruction
of certain mathematical objects.  The middle examples begin to ``kick
back'' -- the blend provides an unexpected axiom, which turns out to
be mathematically provocative.  We noticed also that ``partial
combinations'' are a better reflection of mathematical practice than
``total combinations.''  Finally, the future-oriented challenge
development poses several problems for implementation, but also
suggests that blending, when used at its full power, could offer a
novel approach, relevant to practising mathematicians,
as well as for computer mathematics, and in learning science.

As regards computational creativity, in the current paper we have
included reconstructions, but also shown how computed blends can suggest
new mathematical definitions and problems.  Analysis offered here
shows how this work is a building block that will be useful for future
developments that are able to reason more flexibly about mathematical
problems -- and systematically find and propose  new concepts and problems.

%%% Local Variables: 
%%% mode: latex
%%% TeX-master: "mathsICCC"
%%% End: 
