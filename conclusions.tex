\section{Conclusions}
\label{sec:conc}

We have made the case for dialogicity in CC and in CC research,
building on a comparison of different ways to read poems, and
considering the potential of an existing software system to support
dialogue at various levels.

Our requirements for communication points to the need for program
elements to be able to learn and adapt.  Genetic programming
\cite{koza1992genetic} and related methods offer a certain precedent
for this way of working.

The current paper has focused on more global concerns.  The current
FloWr ``ecosystem'' focuses on features like user interface ease for
human programmers, plus reusability of modules.  The next steps should
involve an agent-based redesign.  We pointed to a considerable amount
of related work in an earlier section.  Off-the-shelf open source
software exists to support some relevant basic features.

However, the question of \emph{context} has not been adequately
addressed either in CC or in mainstream programming.  While we do not
agree with interpretations of artwork that macro-reductionistic in the
sense that ``the environment determines the artwork'' nevertheless
there are important contextual effects \cite{geertz1976art}, and
artwork generally involves an engagement with context.
Philosophically this situates the work in the realms of
\emph{pragmatics} \cite{sep-pragmatics} and \emph{metalinguistics}
\cite{gombert1994development}.  Practically speaking, the questions
relate to building sophistication in the specific ``metalanguage'' of
programming.  The programming tasks involved are not necessarily ends
in themselves, however, but oriented, in the first place towards
building better artefacts.  These too have a goal, which (at least in
a simple case) is to communicate.

The graphical programming environment that shipped with the 1984 robot
simulation game ChipWits, not taken seriously as a competitor to
FloWr, does nevertheless have some useful programming facilities, like
the ability to loop and run different paths in the flowchart depending
on conditions, and the ability for one flowchart to reference another
one, as with spreadsheets.  However, it does not have the ability to
self-program, which is the essence of the proposal for FloWr.  The
ability to generate-and-check (using a population-based mechanism) and
more importantly, the ability to learn over time (as we explored from
a formal perspective in \cite{colton-assessingprogress}) are two other
features that should come standard in future versions of FloWr.  We
want to be able to take account of the abundance of available
information, to introduce ``noise'', and fully take account of the
existing ``framing'' in relationship to the context/situation.

Although social creativity has been explored in CC, it tends to be an
exception rather than the rule.  In future work, we'd like to be able
to say ``We've done it!'' although to develop a concrete proof of
concept we will have to focus on a few simple metrics (like
musicality).

There is also a ``social engineering'' component to this paper, hoping
to motivate a new approach to research evaluation in CC.  Between the
kind of design research carried out in this paper and a large-scale
double-blind study that would ``prove'' the superiority of a social
approach is something more contextual, involving the actual practices
and abilities of participants
\cite[pp. 167--185]{seikkula2006dialogical}.  We sketched an approach
to the evaluation of poems, but similar thinking can help develop an
evaluation of programs.  The question of how much architecture should
be shared in the CC community: do we need a shared ``CCyc'' platform,
or should we ``let a hundred flowers bloom''?  One moderate approach
would be to revive the \emph{floral games} of the troubadours.

%%% Local Variables: 
%%% mode: latex
%%% TeX-master: "mathsICCC"
%%% End: 
