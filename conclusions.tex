\section{Conclusions}
\label{sec:conc}

% BACKGROUND
% How has the social dimension of creativity been explored in CC to date? 
The ideas of social interaction, feedback, and evaluation have frequently been discussed in CC, but implementation
and theorisation around these topics have been more limited.
% PHILOSOPHY AND METHODS
% How can a created artefact tell us about its making, and what can this contribute to CC?
In the current paper, we suggest giving artefacts more agency, designing computer programs with more autonomy, and focusing research effort on creative evolution.
% DESIGN
% How can computer poetry contribute to developing a process-based theory of poetics?
We have shown that in principle computers can engage in dialogue about poems, which points to a theory of poetics rooted in the making of boundary-crossing objects and processes.
% FLOWR
% What would have to change about the FloWr system to implement the computational poetry workshop approach?
In order to move from thought experiment to computational simulation, FloWr could be helpfully extended with further programmer facilities including loops, subroutines, and commentaries, along with the ability to generate-and-test in a population-based manner, and the ability to learn over time.
% DISCUSSION
% What are the pros and cons of the workshop approach?
Workshops and related approaches are suitable for autonomous learning and development of the creative process, but they face technical and also some theoretical limitations.   
% CONCLUSIONS
% What might be the future role of dialogue in CC?
Dialogue may offer a way to creatively push these limits, empowering both programs and programmers.


%%% Local Variables: 
%%% mode: latex
%%% TeX-master: "mathsICCC"
%%% End: 
