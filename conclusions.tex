\section{Conclusions and Remarks}
\label{sec:conc}

The examples presented in this paper trace the development of the
blending approach.
The current paper begins with reconstructions, but also
quickly shows how computed blends can suggest new mathematical
definitions and concepts of interest to practising mathematicians.
The analysis offered here shows that this work is a building block that
will be useful for future developments that are able to reason more
flexibly about mathematical problems -- and systematically find and
propose new concepts and problems.

In future work, we will look more at the cognitive issues raised in
this work. In particular, the use of \emph{image schemas} can give a
link between the computational and representational approach taken
here, and the cognitive claims coming from authors such as Fauconnier
and Turner, and Johnson.  Here the work of \textcite{ManPag14} and
\textcite{HeKuNe14short} gives an idea of how these underlying
cognitive primitives can be expressed in logical form, and can thus
play an explicit role in our modelling of creativity in mathematics.


%%% Local Variables: 
%%% mode: latex
%%% TeX-master: "mathsICCC"
%%% End: 
