\documentclass[letterpaper]{article}
\usepackage{iccc}

\usepackage{dialogue}

\usepackage{etoolbox}

\makeatletter
\appto{\PreDialogue}{\global\@newlistfalse}
\makeatother

\newcommand{\jc}[1]{\speak{Joe}{#1}}
\newcommand{\aj}[1]{\speak{Anna}{#1}}
\newcommand{\tl}[1]{\speak{Teresa}{#1}}
\newcommand{\cg}[1]{\speak{Christian}{#1}}

\usepackage{censor}
\usepackage{enumitem}
\usepackage{times}
\usepackage{helvet}
\usepackage{courier}
\usepackage[T1]{fontenc}
\usepackage{csquotes}
\usepackage[utf8]{inputenc}
\usepackage[british]{babel}
\usepackage[style=apa,natbib,backend=biber,url=false]{biblatex}
\let\cite\citep
\usepackage[ddmmyyyy]{datetime}
\renewcommand*{\multicitedelim}{\addsemicolon\space}%
\usepackage{xcolor}
\definecolor{light-gray}{gray}{0.97}
% \usepackage[firstpage]{draftwatermark}
% \SetWatermarkColor{light-gray}
% \SetWatermarkText{\shortstack{Draft\\\today\\\currenttime}}
% \SetWatermarkScale{0.75}

\usepackage{tikz}
\usetikzlibrary{quotes,decorations.text}
\usepackage{figures/circle_arcs_setup}

\usepackage[framemethod=tikz]{mdframed}
\mdfsetup{
skipabove=\baselineskip,
skipbelow=\baselineskip,
innertopmargin=3pt,
innerbottommargin=5pt
}

\newcommand*{\sourceatright}[1]{\unskip\hspace{1em plus 1fill}%
\nolinebreak[3]\hspace*{\fill}\mbox{#1}}%LaTeX Hacks 

\usepackage{etoolbox}
\makeatletter
\patchcmd{\@verbatim}
  {\verbatim@font}
  {\verbatim@font\footnotesize}
  {}{}
\makeatother

\usepackage[
  bookmarks=false,
  pdfpagelabels=false,
  hyperfootnotes=false,
  hyperindex=false,
  pageanchor=false,
%  colorlinks,
  hidelinks
]{hyperref}

\DeclareLanguageMapping{british}{british-apa}
% \addbibresource{papers.bib}
\addbibresource{poetryICCC.bib} %

%% \urlstyle{tt}
\AtBeginBibliography{\def\UrlFont{\footnotesize\tt}}
\renewcommand*{\finalnamedelim}{%
  \ifnumgreater{\value{liststop}}{2}{\finalandcomma}{}%
  \addspace\&\&\space\space}
%
\AtEveryCitekey{\renewcommand*{\finalnamedelim}{%
  \ifthenelse{\value{listcount}>\maxprtauth}
    {}
    {\ifthenelse{\value{liststop}>2}
       {\finalandcomma\addspace\bibstring{and}\space}
       {\addspace\bibstring{and}\space}}}}

\AtBeginDocument{\renewcommand*\finalandcomma{\addcomma}}
\AtBeginBibliography{%
  \renewcommand*{\finalnamedelim}{%
    \ifthenelse{\value{listcount}>\maxprtauth}
      {}
      {\finalandcomma\addspace\bibstring{and}\space}}}

% \pdfinfo{
% /Title (Test)
% /Subject (Proceedings of ICCC)
% /Author (ICCC)}
% The file iccc.sty is the style file for ICCC proceedings.

\title{Computational Poetry Workshop: Making Sense of Work in Progress\\[.2cm](A play in one act)}
%% \author{J. Corneli\thanks{Corresponding author. Email: {\tt j.corneli@gold.ac.uk}} \\ Goldsmiths College
%% \And A. Jordanous \\ University of Kent \\
%% \And R. Shepperd \\ Goldsmiths College
%% \And \textbf{M. T. Llano}  \\ Goldsmiths College
%% \AND \textbf{J. Misztal}  \\ Jagiellonian University
%% \And \textbf{S. Colton}  \\ Goldsmiths College
%% \And \textbf{C. Guckelsberger}  \\ Goldsmiths College
%% }
\setcounter{secnumdepth}{0}

\thispagestyle{plain}
\pagestyle{plain}

\begin{document} 
\begin{center}
\textbf{Computational Poetry Workshop: Making Sense of Work in Progress}\\[.2cm](A play in one act)
\end{center}

\bigskip

%\maketitle
%% \begin{abstract}
%% \begin{quote}
%% Creativity cannot exist in a vacuum; it develops through feedback,
%% learning, reflection and social interaction with others. However, this perspective
%% has been relatively under-investigated in computational creativity
%% research, which typically examines systems that operate
%% individually. 
%% % We develop a set of requirements for creativity systems
%% % that can incorporate communication and give and receive
%% % feedback. 
%% % As a thought experiment to test these ideas, 
%% We develop a thought experiment showing how structured dialogues can help develop the
%% creative aspects of computer poetry.
%% Centrally in this approach, we ask questions of a poem, inviting it to tell us in what way it may be considered a ``creative making.''

%% % in what way can CC enhance the human workshop?

%% \medskip

%% \textbf{Keywords}: computer poetry, social creativity, flowcharts, Writer's Workshops
%% \end{quote}
%% \end{abstract}

% Study papers
% These will be papers which draw on allied fields such as psychology, philosophy, cognitive science or mathematics;
% or which appeal to broader areas of Artificial Intelligence and Computer Science in general;
% or which appeal to studies of the field of Computational Creativity as a whole.
% The emphasis here is on presenting enlightening novel perspectives related to the building,
% assessment or deployment of systems ranging from autonomously creative systems to creativity support tools.
% Such perspectives can be presented through a variety of approaches including ethnographical studies,
% thought experiments, comparison with studies of human creativity and surveys.


%% TODO LIST FOR REVISION
% - trim the introduction

%% \begin{mdframed}
%% {\small
%% \vspace{-.3cm}
%% \tableofcontents
%% }
%% \end{mdframed}
%% \newpage

%% \begin{quote}
%% {\small `We \emph{can} talk,' said the Tiger-lily: `when there's anybody worth talking to.'\\
%% \sourceatright{Through the Looking Glass, Lewis Carroll}}
%% \end{quote}



\textbf{Dramatis Personae.}

\smallskip

\begin{tabular}{lp{.7\columnwidth}}
\sc{Joe} & Presenting author \\
\sc{Anna} & Reviewer 1 \\
\sc{Teresa} & Reviewer 2 \\
\sc{Christian} & Workshop moderator \\
\end{tabular}

\bigskip

\section*{Initial backdrop}

\medskip

A picture like this, with the title of our paper above and the names
of authors below:

\medskip

\begin{mdframed}
\begin{center}
\textbf{Computational Poetry Workshop:\\ Making Sense of Work in Progress}

\includegraphics[width=\columnwidth]{boink}

J. Corneli, A. Jordanous, R. Shepperd, M. T. Llano,\\J. Misztal, S. Colton, and C. Guckelsberger

\end{center}
\end{mdframed}

\medskip


\section*{Scene 1}

\begin{dialogue}
\direct{The workshop participants have gathered to discuss the paper
  ``Computational Poetry Workshop: Making Sense of Work in Progress.''}

\smallskip

\cg{If you could move these chairs together in a sort of circle\ldots}

\direct{They move the chairs into more of a semi-circle, so that the
  audience can see them too.}

\aj{If we can all see each other that should be fine.}

\cg{I guess you've all had a chance to read the paper already?}

\direct{The reviewers nod and say ``yeah.''}

\cg{Joe, could you give us a quick summary and highlight the things
  you're most interested in getting feedback about?  I guess, given
  the content of the paper, you're pretty much familiar with the way
  this sort of thing works, but, just to be clear, once you're done
  with that quick presentation, we'll give you feedback, and you
  should just take notes. OK?}

\medskip

\direct{Sometime around during the above speech, the title slide
  should go away, and be replaced by a blank white screen.}

\end{dialogue}

\newpage

\begin{dialogue}
\direct{As \refer{Joe} speaks, the pieces of text in italics will appear
  one by one as bullet points on the screen behind the workshop participants.
  That slide stays on screen throughout the rest of Scene 1 (\refer{Joe}'s monologue).}

\begin{mdframed}
\begin{itemize}
\item \emph{How has the social dimension of creativity been
  explored in CC to date?}
\item \emph{Let's give artefacts more agency, design computer programs
  with more autonomy, and focus research effort on understanding
  creative evolution.}
\item \emph{Let's ask: ``How a created artefact can tell us about its own
  making?''}
\item \emph{A theory of poetics rooted in the making of
  boundary-crossing objects and processes.}
\item \emph{The paper is claiming that workshops are suitable
  environments for autonomous learning and development of the creative
  process.}
\end{itemize}
\end{mdframed}

\jc{Sure.  OK, so, the paper is called ``Computational Poetry
  Workshop: Making Sense of Work in Progress.''  It's for ICCC'15, the
  International Conference on Computational Creativity.  It's already
  been accepted for publication but I'd like to improve it so that it's
  a bit more clear, since the reviewers expressed some doubts.  The
  paper starts out by looking at \emph{how the social dimension of
    creativity has been explored in CC to date.}  It basically says
  that the ideas of social interaction, feedback, and evaluation have
  frequently been discussed in CC, but that implementation and
  theorisation around these topics have been more limited.  It then
  goes on to suggest that we what should do is \emph{give artefacts
    more agency, design computer programs with more autonomy, and
    focus research effort on understanding creative evolution.}  I
  want to check with you if that part is clear, because the
  perspective isn't exactly traditional.  The underlying idea is much
  more broadly applicable than CC, and it asks: \emph{``How a created
    artefact can tell us about its own making?''}  The paper takes
  this idea and runs with it a bit, to suggest that in principle
  computers can engage in dialogue, with, and about, poems.  Out of this comes the
  more profound idea of \emph{a theory of poetics rooted in the making
    of boundary-crossing objects and processes.}  That sounds a bit
  abstract, so I want to make sure it's coming across well.  After
  presenting these more theoretical ideas, the paper makes some
  initial steps towards a computational simulation.  It develops a
  thought experiment that imagines an improved version of the FloWr system
  that's being developed in the CCG group at Goldsmiths being used
  to run ``writers workshops'' for computer poetry.  There are lots of
  technical facilities that could make that easier, but the main thing
  I want to get feedback from you on today is what you think about the
  core idea of the workshop approach in a computing context.  \emph{The paper is claiming that
    workshops are suitable environments for autonomous learning and
    development of the creative process.}  Finally, it makes some sort
  of political claims about dialogue within CC, but the basic thrust
  here is that we should be getting our computer programs talking to
  each other.  I'm curious to know what you guys think about that.
  And, yeah, that's it.}

\section*{Scene 2}

\direct{The projection switches to a blank screen again.  \refer{Joe} takes
  notes on what people are saying, and these notes accumulate on the
  screen.  This should resemble a poem by the end of the scene.}

\cg{Thanks Joe.  Does anyone want to sum up what they think the paper
  is about? \direct{Looks around.}  I guess I'll start.  The paper
  studies how social interaction through dialogue can be applied to
  computational creativity processes.  It uses poetry generation as an
  example and aims to develop a ``thought experiment'' in this
  domain.}

\tl{I'd say it's basically a study about how to use social creativity
  to generate poems. The proposal consists on using the Writer’s
  Workshop model to maintain a dialogue between different poetry
  generators and poetry critics in order to improve the final poem.
  In the paper, there are presented different questions that could be
  asked by a person when reading a poem, and the questions that a
  computer could ask.  Then, the results are compared.  Finally, the
  proposal is applied showing a hypothetical adapted design of the
  FloWr system with a human critic and a poetry generator.}

\cg{Thanks Teresa, I think that's a good summary.  You're right, the
  questions that people and computers can ask form a pretty central
  part of the paper.  Maybe that has to do with the ``theory of
  poetics that is rooted in the making of boundary-crossing objects
  and processes'' although frankly that's still pretty abstract for
  me.  Anna, what do you think?  What does the paper accomplish?}

\aj{Well, first of all, I think this paper makes a reasonable start
  toward describing a workshop-style collaboration on writing (and
  reading) poetry.  It does only preliminary work toward this, but
  raises a lot of interesting questions.  For example, it made me
  wonder how a workshop approach differs from more traditional peer
  review -- if at all?  Like Teresa said, it seems like the paper is
  making the case that the workshop is not just useful for reviewing
  things after they're created but for helping create them in the
  first place.  Although there is some interesting tension there.  If
  the emphasis is on poetry generation it would probably be useful to
  include a more thorough review of systems like Slant (by Montfort,
  P\'erez y P\'erez and Harrell), which is a blackboard-based
  multi-agent system.  But I can see that the workshop idea is a bit
  different from more traditional multi-agent systems like that one.}

\tl{I think it's really nice to see the discussion of process as the
  content of a poem, not because everyone would agree that this is
  correct but because it asserts a particular, deliberate perspective
  on poetry.  Christian, I think what the philosophical ideas are
  saying is that the poem is a sort of dialogue that the reader gets
  involved with -- but I agree that could be said a lot more simply.
  Another thing is that the way in which the agency is split up on
  page 4 (a separate counting agent and breathing agent) of course
  does not represent the different perspectives of workshop
  participants.  A workshop where everyone has the same sort of
  expertise, along these lines, could be successful as a computer
  program.  But if it is very culturally homogeneous, it's not going
  to say anything very interesting.  My question is how this model can
  include cultural and idiosyncratic differences in ways that are
  given first-order representations, given how important those
  differences are.}

\cg{OK. Now I think we're getting into the question ``What could be
  improved about the paper?''  So, what do you think about that?}

\tl{To be honest, it is sometimes difficult to read.  I also think a
  better example with more participants could be useful to understand
  the proposal.  In fact, at some point quite a few more examples and
  experiments will really be needed, to really evaluate the proposal,
  but I understand that the current version is a study paper rather
  than a technical paper.}

\aj{Following on from what Teresa said, my only concern here is that
  too little has been done in terms of modelling and system-building
  for others to effectively build upon, or argue for or against the
  applicability of this kind of design in a computing context.  If you
  want to be more computationally convincing without building a
  concrete implementation, I was thinking you could break the process
  into phases like these: \direct{\refer{Anna} refers to her notes,
    and indicates each step clearly, with a gesture or number;
    \refer{Joe} should make sure to capture each of these points in
    his on-screen notes.} \direct{1} Building a model of the text,
  \direct{2} Tagging elements of interest, \direct{3} Generating
  feedback based on assocations drawn between these elements,
  \direct{4} Asking for more information to understand the feedback,
  \direct{5} Creating rationale for the feedback and, on the part of
  the author, \direct{6} Updating the point of view.  To be honest,
  any one of those steps could probably be a PhD project, but with a
  division of labour like that, it starts to look more computationally
  feasible.}

\cg{I also find it a bit hard to see the computational focus in the
  current paper, although maybe that's because it's doing double duty
  as a kind of manifesto.  I think it is certainly presenting a very
  interesting challenge and food for thought for later discussion and
  more precise elaboration.  Oh, and there are also a few typos, I can
  pass you a list. Any other comments?}

\direct{Looks around, but people are signalling they're done.}

\cg{Joe, do you have any questions about the discussion so far?}

\jc{\direct{A bit hurt because he's imagining the philosophical ideas
    haven't come across very well.}  No, thanks, I think that's all
  pretty clear.}

\cg{I think there's a lot there, but -- if I can make a final summary
  comment -- I think it would be good if the work was presented with
  some more structure.  If not in the paper itself, then at the
  conference: a well structured and concrete presentation, perhaps
  following an example that looks into the role of different agents
  could help spark the discussion.  \direct{Breaking the 4th wall.}
  With that, let's take some questions and comments from the
  audience.}

\bigskip

\direct{(Possible) Applause, and Q\&A.}

\end{dialogue}
%% \section{Introduction}
\label{sec:intro}

% \begin{itemize}
% \item Why poetry?
% \item Why FloWr?
% \item Why Workshop?
% \end{itemize}

We are writing in a large part to champion Alan Turing's
proposal that intelligent machines should ``be able to converse with
each other to sharpen their wits'' \cite{turing-intelligent}. 
%
The formalism that we propose builds on the notion of social cybernetics that flows
from the following propositions from Heinz von Foerster, which he uses to theorise systems
in which participants can responibly specify their own roles in relationship to
other system participants:

\begin{quote}
``Anything said is said \emph{by} an observer.'' \\
``Anything said is said \emph{to} an observer.''\\
\sourceatright{\cite{von2003cybernetics}}
\end{quote}

According to Jaako Seikkula and Tom Arnkil, who draw on the philosophical and literary analysis of Mikhail Bakhtin \cite{bakhtin2010toward,bakhtin1984problems} in their dialogical approach to psychosocial work,
\begin{quote}
``Dialogues could be called `the art of crossing boundaries'.  Instead of trying to control others, the parties reach out towards each other to hear their views better, to generate shared languages and to join resources.''
\sourceatright{\cite[p. 23]{seikkula2014open}}
\end{quote}

This paper outlines a study of social creativity with a dialogical emphasis, taking computer
poetry as our working domain.  It uses the Writer's Workshop model
\cite{gabriel2002writer} as the virtual laboratory in which to conduct a thought experiment.
The findings of our study are applied to the FloWr system \cite{charnley2014flowr}. 
We focus on the following questions in turn:
%\subsection{Outline}


\begin{itemize}
% BACKGROUND
% - a few more details
\item How has the social dimension of creativity been explored in CC to date? 
% PHILOSOPHY AND METHODS
% - break into subsections and add tools, integrate to table
%CG: Modified this to connect rosies methodology to the workshop and CC
\item How can a created artefact tell us about its making, and what can this contribute to CC?
% DESIGN
\item How can computer poetry contribute to developing a process-based theory of poetics? 
% FLOWR
% - Teresa to revise 
\item What would have to change about the FloWr system to implement the computational poetry workshop approach?
% DISCUSSION
% - keep it?
\item What are the pros and cons of the workshop approach?
% CONCLUSIONS
% - taking into account Christian's comment, I've revised this and will edit the conclusion to align.  JC.
\item What might be the future role of dialogue in CC?
\end{itemize}

%%% Not nec. needed
%%% In the background section, we will focus on social creativity in the computational creativity domain and more specifically in computer poetry. This will be followed by a Methods section, in which we ...

%%% Local Variables: 
%%% mode: latex
%%% TeX-master: "poetryICCC"
%%% End: 
 
%% \section{Background}
\label{background}

\subsection{Social creativity in CC}

Although we have adopted the term ``social creativity'' following
\cite{saunders12} and with the specific understanding developed above,
we could also refer in a similar spirit to situated, interactive,
communal, contextual, conversational, group, dialogical,
discourse-based, community-based, interaction-based, or
feedback-informed creativity.

The point is that creativity cannot exist in a vacuum.  The very
essence of creativity lives in its appreciation by the creative entity
themselves and its audience.  As we have remarked elsewhere,
creativity is in the eye of the beholder.  During the creative
process, self-evaluation abilities are crucial
\cite{poincare29,csik88}. Social creativity expands upon this
paradigm by bringing co-creators into the process, and creating works
that rely on dialogue, reflection, and multiple perspectives. The
``results'' may be steeped in process and will not always be based in
consensus.

The Four Ps of creativity -- the creative Person, Product, Process and
Press (i.e. environment) \cite{rhodes61,mackinnon70} -- have been
emphasised in general creativity research.  \emph{Pluralising} these
terms would call attention to a social dimension of creativity, and
leads to a more inclusive and encompassing approach to the study of
creativity -- one that accommodates multiple perspectives. The
Pluralised Ps remind us that it is not sufficient to model a lone
creator or to generate an attractive artefact.  To model creativity
more completely, we also need to consider the environment in which a
creative person operates, and how the environment is used in the
creative process.

Computational creativity research has achieved many successes in
computational generation of creative products. The question of how
these systems could adapt and learn from feedback to improve their
creativity, however, remains underexplored in computational creativity
despite evaluation being a pivotal contributory part of the creative
process. Researchers have generally preferred to take on the task of
generating artefacts that could be seen as creative, as a necessary
prior to the task of incorporating self-evaluation within a creative
system \cite{jordanous11iccc}.

Some notable exceptions exist, highlighting the importance of the
environment in which a creative system is situated \cite{mcgraw93,
  sosa09, perezyperez10MM, pease10, saunders12}, with some of this
work influenced by the DIFI (Domain-Individual-Field-Interaction)
framework \cite{csik88}. Generally, however, social interaction
between creative agents and their audience is an area which has been
neglected. Increased development of the interactivity of creative
systems, especially where this affects the way these systems works, is
pleasing to see and deserves further attention
\cite{coltonwiggins12}.

\begin{mdframed}
\textbf{Other CC material?  Anything about poetry in particular?  It
  might be a good idea to mention Tristan Tzara, Brion Gysin, etc.
  Probably at least a few words about FloWr here as well.}

 E.g. \cite{jordanous10}?  Definitely we should cite
\cite{misztal2014poetry} and explain a bit about the MASTER system.
\end{mdframed}

\subsection{Writer's Workshops}

Quoting \cite[pp. 2--3]{gabriel2002writer}:

\begin{quote}
The original idea behind the writers' workshop was to do a \emph{close
  reading} of a work, to use the term F. R. Leavis coined for the
practice of looking at the words on the page rather than the
intentions of the author or the historical and aesthetic context of
the work.  Under this philosophy, the workshop doesn't care much what
the author feels about what he or she wrote, only what's on the page.
This corresponds to the philosophy of the New Critics, which held that
the work was its own ``being,'' with its own internal consistency and
coherence, which could be studied apart from the author.  Moreover,
this approach is nearly identical to that of the Russian formalists,
who thought that the proper approach to literature was to study how
literary texts actually worked, their structures and devices.
\end{quote}

Framing and any other contextualisation of the work \emph{as it is
  intended to be presented} is permitted, and receives critical attention.

In \cite{serendipity-arxiv}, we described a template for a pattern
language for interactions in a computational poetry workshop, closely
following Gabriel's outline of the relevant steps: {\tt presentation},
{\tt listening}, {\tt feedback}, {\tt questions}, and {\tt
  reflections}.\footnote{To this should be added the potential for
  real-time {\tt replies} by critics to {\tt questions} asked by the
  presenting author, before subsequent ``offline'' {\tt reflections}.}
We used this template to expand several of the patterns of serendipity
described by \cite{van1994anatomy}, showing how they could be used to
foster discovery and invention in a workshop environment.



%%% Local Variables: 
%%% mode: latex
%%% TeX-master: "poetryICCC"
%%% End: 
 
%% \section{Philosophy and methods}
\label{sec:philosophy_and_methods}

% Q.
Why do we need a model that might enable us to observe the
creative process in the artistic outcome or practice e.g poems? Or,
how can we justify building such a model based on the outcomes of
creative practice i.e poems?

% A.

1. Because the originary and therefore the unpredetermined nature of
the creative process means that the post-fact outcome represents a
more accurate and objective evidence of it than the poet's attempt to
explain it as it happens.  According to Kant, the creation of a work
of art (e.g poetry) succeeds in exhibiting originality that is neither
predictable before it occurs nor traceable to prior rules
\cite{anderson1992role}.  So a creator discovers how and what he/she
is expressing through the creative process only in the course of doing
it. And we, as observers, are therefore only able to consider how a
creator selected and rejected various possibilities during the
creative process, by considering the creation after the fact and
within the finished whole.

2. Making a model of the creative process incorporates Dewey's ideas
that the content of the art form is not the same as the aesthetic
emotion expressed in it \cite[p. 35]{dewey2005art}. It is a reasonable
extrapolation from this notion to the idea that the content of an art
form represents and indicates an examination of how it was made.

% Q
If content does not represent conceit, metaphor or what
might be termed raw ingredients or some emotional spark felt the
artist, how can a new way of considering content result in a rigorous
and analytical study of the creative process?

% A
Examining the content of the poem is a way of examining what it tells
us about the drivers propelling it through its own making and how it
stands as observer of its own process.  ``The painter does not paint;
he watches himself paint'' \cite[p. 7]{collingwood1958principles} ``In a poem, objective
material becomes the content and the matter of the emotion and not
just its evocative occasion'' \cite[p. 69]{dewey2005art}.

% Q
What is the central plank to this way of linking the outcome (or
creative practice) with the creative process?

% A
The idea that creation involves an exploration and expression of an
aesthetic emotion and that both these can be charted or mapped in the
creative outcome (the poem).

% Q
What is being expressed?

% A
The artist, in the course of creating, is involved in something
Collingwood referred to as ``the expression of aeasthetic emotion''
\cite[p. 117]{collingwood1958principles}. In Part I of his Principles
of Art, Collingwood suggests that the creative process takes place in
stages. These are not necessarily chronological but present as a
manifestation during creation that is visible in its outcome.

% Q
What do we mean by the ``expression of an aesthetic emotion''?

% A
I refer you to Collingwood and also to Benedetto Croce here for a
combined definition of how an aesthetic emotion is expressed - An
aesthetic emotion is expressed via a total imaginative activity. This
is taken from Collingwood's Principles (1938) and Croce's Aesthetic
(1902).  It is be relevant to the analysis of the poems that Croce's
  Aesthetic defined intuition as ``the non-divisible expression of
  sympathy'' \cite[pp. 171--193]{kemp2003croce}.

% Q
What are the steps, stages or aspects of the creative process?
[``Aspect'' is my preferred term as it removes us from chronology and
  the implication that all these must be present all the time. Think
  instead that they can be ``evidenced''].

% A
Collingwood describes the initial stages of expression as
``oppression''; as something that happens to the artist during his/her
exploration of the creative process and, unexpressed, this produces
feelings that are oppressive and which Dewey described as
``disturbance'' and which Anderson and Hausman see as a ``colouring''.

% Q
What happens to this oppression or disturbance?

% A
The artist becomes conscious of it and starts to explore his/her own
expression of it. This takes place as an intuited feeling (as
described by Croce). This gives rise to a new feeling of alleviation
or easement. That reference to novelty is crucial. If something is
new, it cannot be predetermined.

% Q
What do we mean by ``aesthetic emotion''?

% A
PG Whitehouse on Dewey's Art as Experience suggests that Dewey joins
Collingwood in separating aesthetic emotion from any notion that
inspiration can be considered as something like raw materials. This
fits in with our view of content as representational (and certainly
indicative) of the creative process of the poet. So an emotion is
aesthetic when it ``adheres to an object formed by an expressive act''
\cite[pp. 149--156]{whitehouse1978meaning}.  Or better still, ``the art
object does not have emotion for its significant content\ldots.Emotion
is a conscious sign of a break, actual or impending. It belongs to the
self that is concerned in the movement of events toward an issue that
is desired or disliked'' \cite[p. 14]{dewey2005art}.

% Q
What do we mean by experience?

% A
``An unanalysed whole in a situation, having a pervasive quality in
which the self finds itself'' \cite[p. 3]{zeltner1975john}.

% Q
In what way (in what phases/stages/aspects) can we see the artist
experiencing aesthetic emotion in the making of his/her art?

% A
In a study of Collngwood's theory of the expression of aesthetic
emotion, Doug Anderson and Carl Hausman refine ideas on how we might
see this and specifically how this relates to our way of studying
process through practice \cite[pp. 299-305]{anderson1992role}:

\begin{quote}
Aesthetic emotion\ldots  response\ldots  artist's decision on components of expression\ldots feeling of easement plus a simultaneous emerging of a unique imaginative expression\ldots alleviation\ldots consciousness\ldots specific to converting psychical emotion\ldots unique aesthetic experience 
\end{quote}

% Q
What is psychical emotion?

% A
The agent (through expression) discovers him/herself to have been
feeling independently of expressing it. [The emotion of consciousness
  is where an agent only feels at all in so far as he/she thus
  expresses it].

% Q
How is easement understood to be unique/originary?

\begin{quote}
Aesthetic emotion\ldots attends to successful expression\ldots contributes integrally to what is expressed in its specificity\ldots functions as individualsed clue\ldots at each crucial moment of a developing imaginative experience
\end{quote}

Conclusion: The poem is a work of progress, rather than a work in
progress.

Although it is beyond the scope of the current work to trace these
connections, we will remark that this view is connected with the
Bergson genre of philosophy, running to Mead who offers a generalised
view of the social, to Bakhtin who develops the notion of dialogue in
a broad metalinguistic frame, and to Deleuze who develops a processual
ontology based on the idea of difference \cite{bergson1983creative,
  mead1932philosophy, bakhtin1984problems, deleuze1994difference}.
These perspectives are relevant to the interest we take here in
emergence, polyvocality, and learning by specifying and engaging with
problems.

\subsection{Methods: ``What are the proposed `lab rats'?''}
\label{sec:methods}

\begin{figure}
\resizebox{\columnwidth}{!}{%
%% For final:
\begin{tikzpicture}
%  \draw[thick] (4,0) node[below={2cm}] {\emph{A.~``mere generation''}}
%                     pic[red, -latex]{darc=100:270:1.3cm:Chicken:Lay:2.4cm:3cm}
%                     pic[red, -latex]{uarc=280:450:1.3cm:Egg:Hatch:1cm:.1cm};
%
  \draw[thick] (4.0,0) node[below={2cm},align=center] {\emph{A.~how to become a writer\footnotemark}}
                      pic[red, -latex]{darc=100:270:1.3cm:{}:{Write for 8 hours a day}:1cm:.3cm}
                      pic[red, -latex]{uarc=280:450:1.3cm:{}:{Read for 8 hours a day}:1cm:.3cm};
%
  \draw[thick](8.5,0) node[below={2cm}] {\emph{B.~``the Other''}}
                     pic[red, -latex]{darc=100:270:1.3cm:Statement:Speak:2.1cm:2.6cm}
                     pic[red, -latex]{uarc=280:450:1.3cm:Response:{Listens and interprets}:.2cm:.1cm};

%  \draw[thick] (4,-5.5) node[below={2.2cm}] {\emph{D.~learning by doing}}
%                        pic[red, -latex]{darc=100:240:1.5cm:Poet:Write:3cm:2.7cm}
%                        pic[red, -latex]{rarc=250:320:1.5cm:Poem:Responds:1cm:.5cm}
%                        pic[red, -latex]{uarc=330:450:1.5cm:Context:Speaks:1cm:.1cm};
%
%
  \draw[thick] (13,0) node[below={2cm}] {\emph{C.~proto-workshop}}
                        pic[red, -latex]{darc=100:180:1.3cm:Poet:Writes:1cm:.7cm}
                        pic[red, -latex]{rarc=190:270:1.3cm:Poem:Responds:1cm:.4cm}
                        pic[red, -latex]{rarc=280:360:1.3cm:Context:Speaks:.7cm:.7cm}
                        pic[red, -latex]{larc=370:450:1.3cm:Reader:Responds:.8cm:.2cm};

%  \draw[thick] (13,-5.5) node[below={2.2cm}] {\emph{F.~how computers work}}
%                       pic[red, -latex]{darc=100:180:1.5cm:{}:Print:1.5cm:1.4cm}
%                       pic[red, -latex]{rarc=190:270:1.5cm:{}:Loop:1cm:1.4cm}
%                       pic[red, -latex]{rarc=280:360:1.5cm:{}:Read:1cm:1.4cm}
%                       pic[red, -latex]{larc=370:450:1.5cm:{}:Eval:1.4cm:.2cm};
% Note:
% controls at end: offset of outer text, followed by offset of inner text
\end{tikzpicture}

%% For quick compiling
% \includegraphics{figures/cycles_shortcut}
}

\caption{Several illustrative cycles \label{fig:cycles}}
\end{figure}


There are many possible places for a ``dialogical'' intervention
within the writing process; see Figure \ref{fig:cycles}.  Figure
\ref{fig:cycles}(A) shows the standard chicken-and-egg problem,
designed to provoke questions about \emph{evolution}; 1(B) shows an
analogous picture that gives a simple recipe for the growth and
development of a writer; 1(C) is a formally similar diagram that
squares up to the metalinguistic features of the situation, showing
that a \emph{response} (which may be verbal, visceral, physical or
something else) always has dimensions that goes beyond the utterance
that is overheard; 1(D) examines in further detail what happens when
someone \emph{writes} -- namely, writing as a response to a situation
that may allow the writer to make sense of this situation; 1(E) adds a
\emph{critic} who responds to the situation, particularly as it is
expressed through and enhanced by the poem; 1(F) shows that this
scenario is not as unfamiliar to computer programmers as it might
otherwise sound -- consider that the ``Eval'' phase in a
Read-Eval-Print loop can be interrupted with a debugger to fine-tune
program operation.

Our ``lab rats'' are, accordingly, not poems -- which could, after
all, be developed through ``mere generation'' -- but are, rather,
\emph{instances of reading and responding to poetry}.  This may take
place within a formal ``workshop'' context or they may take place in
smaller-scale experiments where a computer system reads and responds
to other poets.  Naturally, such responses can also be more or less
``canned'' (as with Michael Cook's humorously nonspecific
AppreciationBot), so the question becomes what constitutes an
authentic, interesting, or useful response, and how will these be
developed?  The idea of responses is also useful at the micro-level,
as will be made clear in the following section.  We focus here on the
big picture of staging a encounter, supported by preliminary
implementation work.

%%% Local Variables: 
%%% mode: latex
%%% TeX-master: "poetryICCC"
%%% End: 
 
%% \section{Design} \label{sec:design}

%% See poetry_questions_full.tex for a four-column version
\begin{table}[t]
{\small
\def\arraystretch{1.1}
\begin{tabular}{p{1.8in}p{1in}}
\textbf{Question} & \textbf{Agent concerned} \\[.1cm]
\textbf{\emph{Word level}} & \\
What does this word mean? & WordNet expert \\
%%%%%%%%%%%%%%%%%%%%%%%%%%%%%%%%%%%%%%%%%%%%%%%%%%%%%%%%%%%%%%%%%%%%%%%%%%%%%%%%
Where does this word come from? & Provenance expert \\[.5cm]
%%%%%%%%%%%%%%%%%%%%%%%%%%%%%%%%%%%%%%%%%%%%%%%%%%%%%%%%%%%%%%%%%%%%%%%%%%%%%%%%
\textbf{\emph{Phrase level}} & \\
What is this line about? & Keywords expert \\[.5cm]
%%%%%%%%%%%%%%%%%%%%%%%%%%%%%%%%%%%%%%%%%%%%%%%%%%%%%%%%%%%%%%%%%%%%%%%%%%%%%%%%
\textbf{\emph{Line level}} & \\
What is the sentiment of this line? & Sentiment expert\\
Is this alliteration, rhyme, consonance, etc.~important? & Style expert \\[.5cm]
%%%%%%%%%%%%%%%%%%%%%%%%%%%%%%%%%%%%%%%%%%%%%%%%%%%%%%%%%%%%%%%%%%%%%%%%%%%%%%%%
\textbf{\emph{Poem level}} & \\
How many rhymes are in the poem? & Rhymes expert \\
How well does the poem's metrical structure flow? & Rhythm expert \\
How repetitive is the poem? & Repetition expert
\end{tabular}
}
\caption{Questions we could implement using various computational agents\label{tab:questions_for_computational_agents}}
\end{table}

\subsection{Questions to ask when reading poems}

Table \ref{tab:questions_for_computational_agents} contains a list of
questions that could be addressed, in a programmatic manner, to
analyse an individual poem.  These questions, coming from a computer
poetry perspective, do not make the same assumptions about linguistic
understanding or embodied experience that apply to human poets.  It is
worthwhile to compare this list with a list of questions that a
reasonably sophisticated poetry reader might ask about poems (Table
\ref{tab:questions_for_human_readers}).  The difference between these
two frames of reference is most instructive.

In the approach to ``sophisticated'' reading considered here, we focus
on the process-oriented question: ``What does the poem tell us about
how it was made?''  This can be studied through the lens of the
follow-up question ``What are the aspects of the poem and how are they
present to the reader?''

%% See poetry_questions_full.tex for a narrative treatment of these questions

%% \begin{mdframed}
%% \textbf{I've compressed this down a lot.  Add some references and
%%   expand this a bit.  Our old notes may be useful.}

%% Among the several aspects that may be present in the poem are
%% register(s); addressee(s); position(s); the changing awareness of the
%% reader; character(s); image(s); functions, mechanics, and paradigms;
%% problems, discomforts, and dis-easements; apparent versus
%% substantiated truths; overlapping scenarios; chronology of reading and
%% chronological references; lexical choices that carry a deeper semantic
%% meaning; allusive effects; and manifest uncertainty in the poem's
%% construction.
%% \end{mdframed}

\begin{table}[ht!]
{\small
\def\arraystretch{1.2}
\begin{tabular}{p{1.8in}p{1.1in}}
\textbf{Question} & \textbf{Examples} \\[.1cm]
What is are the register(s) of the poem? & cliched, instructive, imperative\\
Who is addressed? & friend, rival, lover, confidante, pupil\\
What position(s) are present in the poem? & pleading, remonstrating, ephemeral\\
What becomes of the reader whilst reading the poem? & alienated, perplexed, amused\\
Who are the characters in the poem? & ``the falconer'', ``you'', narrator, ``two men'' \\
What is the role of image(s) in the poem? & ``the sea'', ``a bicycle''; multiple meanings\\
What functions, mechanics, and paradigms are present for the reader to engage with? & communication, subverted cliche\\
What problems, discomforts, or dis-easements are invoked in the poem? & horror, self-loathing, rejection, desire \\
How do these evolve? & E.g. an image may start to take over from a register \\
What \emph{is} in the world of the poem as compared with what you only think is? & ``Surely'', ``must''; sacred vs mundane; perspectival vs surreal \\
What are the overlaps, transitions, implicit dialogues? & ``twinned'' lines/ideas, juxtaposed parts of the poem\\
What role does the chronology of reading play, versus references to chronology and chronological positions within the poem? & flagged development, evolution, movement, stasis\\
How are lexical categories used? & flighty adverbs, solid nouns, tortuous adjectives\\
Are there discernible allusive effects? & illustrating the literary apprenticeship of the author (or reader)\\
How does Keats' idea of negative capability feature in the composition -- ``that is when man is capable of being in uncertainties''? & we must worry about overconfidence, over-determined lines\\
\end{tabular}
}

\caption{Questions that we actually ask when reading a poem\label{tab:questions_for_human_readers}}
\end{table}

One of the striking things about Table
\ref{tab:questions_for_human_readers} is that it does not easily
divide itself into ``levels'' in the same way as the items in Table
\ref{tab:questions_for_computational_agents} do.  For instance, even
though lexical effects clearly have to do with an analysis taking
place at the ``word level'' the task that a given selection of words
performs is typically global; that is, it is meaningful at the ``poem
level.''  The questions in Table \ref{tab:questions_for_human_readers}
may themselves be divided into registers and positions.  The process
of reading a poem is also a process of \emph{poiesis}.  Each of the
examples listed in the right-hand column of this table (and a plethora
that are not listed) plays a role in the ``society of mind'' of a
reader, analogous to the agents in Table
\ref{tab:questions_for_computational_agents}.

\subsection{Bridges between `theory' and `practice'}

There are things we can actually point to in poems, and matching
concepts in aesthetic philosophy.  Furthermore, we can actually
approach this scientifically.  The computational approach to poetry is
one way to build the practice/theory bridge.  In particular, our
\emph{ansatz} is that the workshop could serve as a way to deepen an
understanding of a poem's semantics!

There are certain prerequisites.  In addition to the presentation of
written work, an underlying situation is assumed, one that is shared
(with respect to differing points of view) by the poet and the critic
(see Figure \ref{fig:cycles}).  The poet and critic are assumed to
have relatively stable, enduring but evolving, identities -- so that
it would be possible at a given juncture for either one to consider
the question ``Who am \emph{I}?''  and ``Who are \emph{you}?''
\cite[p. 251]{bakhtin1984problems}.

To begin with, in response to a given computer-generated poem:

{\itshape
\begin{verse}
%Demon dog\\[\baselineskip]
%
Oh dog the mysterious demon\\
Why do you feel startle of attention?\\
Oh demon the lonely encounter\\
ghostly elusive ruler\\
Oh encounter the horrible glimpse\\
helpless introspective consciousness\\
\end{verse}
}

A human critic might offer the following feedback:

\begin{quotation}
~\vspace{-1\baselineskip}
\begin{enumerate}
\item The use of the word \emph{mysterious} in the first line has no
  resolution, real or attempted, or quest to find one.
%
\item The use of the word \emph{attention} is not being interrogated
  or acknowledged for its importance.  Its qualifying word is
  \emph{startle}, used here as an adjective; acknowledging the fact
  that the attention is noted but not yet part of the transformative
  of the poem.
%
\item This is repeated in the next references to the aesthetic
  experience as a \emph{lonely encounter} and an \emph{exclusive
    ruler}, which is then qualified as \emph{a horrible glimpse}.
%
\item So reference to the contact made between the poem and its own
  event are made through the words \emph{demon}, \emph{encounter} and
  \emph{consciousness} and all of these are qualified in negative
  terms.
%
\item This poem does not welcome the intimacy of bringing anything to
  aesthetic consciousness so that it might be expressed. Why do I say
  that? Because the words are generalised and \emph{horribly}
  imprecise.
%
\item The really interesting thing about it is its own apparent
  understanding that this is so.  Look at the words \emph{mysterious},
  \emph{feel} \emph{lonely}, \emph{elusive}, \emph{horrible},
  \emph{glimpse}, \emph{helpless}, \emph{introspective},
  \emph{conscious}. They are unspecific -- nothing has been explored
  as the poem moves toward a better understanding of these ideas. They
  describe but they do not illuminate by becoming anything else. They
  all associate exploration with fear and isolation and this is
  (paradoxically) quite an interesting acknowledgment of the poem’s
  refusal to go anywhere i.e become a thing transformed by a creative
  process.
\end{enumerate}
\end{quotation}

Each of these comments is \emph{dual-voiced} in the sense that the
critic is relaying the poet's speech with a new emphasis.  Each such
statement is one side of a micro-dialogue
\cite[p. 73]{bakhtin1984problems}.  The challenge is, of course, to
bring the observations into the awareness of the computer poet, across
the ``digital/analogue divide.''

From a programmer's standpoint, this involves massaging each of the
observations into a language that the computer can understand -- the
most obvious candidate being the programming language the computer
used to generate the poem.  But care should be taken not to just
blythely program the computer with more rules, but rather to give
attention to the process of learning new rules contextually.

Rather more briefly, let us consider a reversal of roles, and put the
computer in the position of critic, looking at a passage from an
historical piece of poetry.  We have selected one that might have --
but in fact did not -- serve as a model for the poem generated above.

{\itshape
\begin{verse}
I'm truly sorry man's dominion\\
Has broken Nature's social union,\\
An' justifies that ill opinion\\
Which makes thee startle\\
At me, thy poor, earth born companion\\
An' fellow mortal!\\
\end{verse}
}

Naturally, the first problem is for the computer to \emph{read} the
poem.  There are various possible approaches to this problem.  One of
the approaches that is most appealing from our point of view is the
automatic generation of a semantic network from the input text
\cite{harrington2007asknet}.  This, again, could be enhanced by
additional meanings that are not just factual, say, but aesthetic
(drawing on the notions informing Table
\ref{tab:questions_for_human_readers}).

We do not propose that bridging in either direction is entirely easy,
but both tasks are entirely feasible.


\begin{figure}
\includegraphics[width=\columnwidth,trim = 0mm 0mm 2mm 0mm,clip=true]{figures/workshop-diagram}
\caption{Schematic design for a workshop built in the FloWr system}
\end{figure}

\subsection{Prototyping a workshop in FloWr}
Development plan (Christian):
\begin{itemize}
	\item Define workshop, including product properties, e.g. variety, innovation, etc. Alternatively, refer to prior definition earlier in the paper and point out benefits again.
	\item Introduce different roles in a workshop: master, working student, criticising student (anything else?). These roles can be combined into more complex ones. 
	\item Introduce tasks they perform: introducing and demonstrating new tools, using tools to generate, analyse to improve creation, etc.
	\item Show the demon dog flowchart as part of the FloWr system, and highlight which roles the different parts represent, and which tasks are present.
	\begin{itemize}
		\item A flowchart could be identified as a student that both creates and asseses (fitness functions) at the same time.
		\item The script/moderator could be identified as the master who guides the student, tells it what to do in which way.
		\item The set of all process nodes can be understood as a toolbox the master can access, to hand tools to the student.
	\end{itemize}
	\item Show what's missing to fulfill the prior definition of a workshop
	\begin{itemize}
		\item Multiple students that collaborate and help improving each other's creations.
		\item Master as mediator between two students, asking one student what tools it uses, and telling the other to adapt them.
		\item The moderator can introduce new tools to the workshop.
		\item Students must be able to notice at one step of the creation process (e.g. while using one tool), that the usage of a prior tool could be improved. They could then either first finish or rewind.
		\item Allow for more than one agent. Allow for more than one agent to work at a time.
	\end{itemize}
\item Show which features are required to realize that:
	\begin{itemize}
	\item Understanding the workshop as a hierachical multi-agent system, where the master is at the top and the students are at the bottom of the hierarchy. 
	\item Simplification by separating flowcharts that create and assess from those which only assess. 
	\item Thus introducing external assesment.
	\item Thus communication between flowcharts required.
	\item Enable an agent (more general, e.g. a student) to explain what tools it is using, in which order, and which parts of the output they affect.
	\item Allow other agents to improve their flowchart by means of this information, e.g. add a particular tool in a particular step.
	\item Enable master as mediator.
	\item Allow master to map observations of poems into new tools, that are made available to the workshop.
	\item Allow communication between process nodes, i.e. tools: E.g. assessment in rhyming node shows that too few words/sentences available for good results. Thus necessary to fetch more/different sentences in a prior step. Problem: agency when using the term "tool" - separates assessment by the agent from using the tool for creation.
	\item Introduce parallelisation, to allow multiple agents to work at the same time.
	\end{itemize}
\item Open questions:
\begin{itemize}
	\item Decision making: It's probably more important to answer how decisions are made to adapt another tool, etc., than stating where they're made. E.g. decision making by means of a fitness function of the current artefact, by means of curiosity, etc.
	\item Add more implementation suggestions
	\item Would it be better to focus on one new aspect of the workshop, and leave the others out? E.g. add master who can introduce new tools, or introduce multiple students that can communicate. One paper might be not enough to get from current FloWr to a full-fledged workshop.
\end{itemize}
\end{itemize}


\begin{mdframed}
\textbf{What's the development plan look like?  Should we show the
  flowchart for the ``demon dog'' poem as a very early prototype and
  concrete example to critique?  We could illustrate -- in principle
  -- how it would be improved in a workshop.}

We'll need to discuss the different levels of the design.  E.g.

Moderator.  

Flowchart A: ``I do this'' (service advertising what facilities it has available: if we can swap them in and out, we could do some kind of A/B test to check whether swapping in a different node actually improves the result according to some critic or other).

Flowchart B: ``I would like to be able to do that too.''
Temporality. More feedback from critical agents.  Learn and adapt the script.

Where is the decision made to modify (e.g. only by the Moderator, or at the level of individual flowcharts, or nodes, etc.?)

This is relevant to the issue of ``parallel solutions''.  At the level of some downstream process: How do we decide which of the parallel solutions to pick and carry ahead to the next step?  Do we give feedback to the previous step in the process?

If we are allowed to take multiple different paths, not just parallel copies of the same process, where is the decision made as to which path to take?

Central control
Distributed control
Autonomous
Global wiring

Critic: ``This `Twitter' node is not good.''

Follow the idea of the prepared mind, add any problem that is noticed to a list of ``problems to solve.''  Again, the log of suggestions/outstanding problems could exist at different levels/timescales.  (NB. And a similar approach can apply for ``questions'' as well as ``problems.''  This ``log'' doesn't just have to be a list, it could also be a frame or flowchart etc.)

Protocol:

flowchart to flowchart?

Do we also need some modification for node-to-node communication?
Alternatively, are we going to follow Christian's suggestion and merge flowcharts and nodes into one kind of entity?

Also, do we need an overall ``protocol'' for the Workshop demo
application, different from the more general communication protocol
that is used?

There will be three different type of messages: questions, answers and
suggestions.

Questions can be about sources of information; e.g. files, online
articles, input from another node, etc., about elements of poems;
e.g. similes, rhymes, keywords, sentiment, etc.; about specific
details; e.g. count, purpose, etc. Answers would be associated to
previous questions, and suggestions are changes proposed by one system
to the other.

Commentary is needed in order to enable the dialogue between
flowcharts. Each node in FloWr has two main components, a set of input
parameters and a set of output variables.  Parameters can either be
sources of information; e.g. the Guardian newspaper, or conditions;
e.g. range of dates. Variables are outputs of different types.  A
commentary would be added to each of them in order to facilitate
communication between flowcharts and nodes. This will enable the
dialogue between the participants in the workshop; i.e. ask and answer
questions, as well as it will allow to identify where to apply
suggestions for new versions of a flowchart as suggested in a workshop
session. The proposed commentary is presented in an Appendix.

%% These were some potential criticisms that seem mostly directed at this level

Do we consider e.g. genetic algorithms present (i.e. generate) several
options and allow the downstream nodes to comment on each one high
level feedback (on the whole generated poem) and micro-feedback (at
the level of individual nodes)

Atomic nodes vs composite nodes (flowcharts).  Can we just forget
about what level we are at?  Node or flowchart -- and use a recursive
approach that can be applied to atomic or composite nodes?
(Christian’s diagram of the two different kinds of communication --
between flowcharts or between nodes -- reduce to one kind of
communication if we take the recursion approach.  Similarly,
reification then becomes possible.  In itself this isn’t a huge
technical advance, but it might allow us to deal with examples of
``morphogenesis'', which is interesting.  E.g. grow until I have 100
process nodes and then stop. That said, ``reification'' is a bit
frowned upon -- but provenance is always a good thing.)
\end{mdframed}



%%% Local Variables: 
%%% mode: latex
%%% TeX-master: "mathsICCC"
%%% End: 

%% \section{Discussion} \label{sec:discussion}

\subsection{Potential applications}

Eventually we would like to see the paradigm advanced here in effect
across CC.  This doesn't mean that we would remove the ``generation''
aspects of CC, but that we would pair them more closely with
reflection.  The workshop method of critique may shift to more closely
model an \emph{atelier} method of creation.  The same skills that
support learning in a writers workshop may support a form of dialogue
with the work itself, leading to richer creative artefacts that show
us more about the creative process.

The workshop brings a range of practical and philosophical challenges
for CC, and the broader field of AI, related to asking and answering
novel questions, the practicalities of learning over time, a sense of
identity and personal style.

Focusing on these questions does not in any way suggest that we should
devalue works from lone creatives, but it does suggest that we think
about how we knit individuals together in the social fabric of the CC
community.  The current model at the International Conference for
Computational Creativity (ICCC) is similar to many other academic
conferences, even though our subject matter is really quite different.
It's all well and good for us to travel and present \emph{our} work to
one another and build \emph{our} sense of community in that way, but
what about a track for computers to present their work?

As \cite{turing-intelligent} foresaw, computers have gotten quite good
at Chess and reasonably good at Go, using methods very similar to the
workshop.  Should we not follow their lead?  Poetry seems a natural
next step; prose literature may be approachable through similar
methods.  Indeed, why bother writing dialogue for a
NaNoGenMo\footnote{\url{https://github.com/dariusk/NaNoGenMo}} novel,
if you could simulate it?

\subsection{Potential criticisms}

One class of criticisms could relate to the appropriateness of
dialogue per se: ``Why not put everything into one flow chart?  Or one
node, for that matter?''  In cases where dialogue is indeed seen to be
necessary, a different sort of question arises, namely, how do we know
if we're doing it well?  E.g.~how will we avoid the pitfalls of
``design by committee''?

Regarding the first set of questions, following Bakhtin, programs that
don't need to be explicitly social can continue as they are.  The
second set of questions will have to be worked out in practice, but we
should keep in mind the potentially greater pitfalls of asocial
design.  Indeed, we wonder if apparent failures of social creativity
are often due to a poor grasp of the social rather than an overly
social approach.

Another line of questioning that we may expect from some CC
practitioners is as follows.  Given the historical emphasis on
\emph{creating} new artefacts in CC, shifting the emphasis to the
computational \emph{appreciation} of already-created artefacts is
somewhat strange.  More pointedly, concerned members of the public may
say: computers creating art is bad enough!  Surely, you don't expect
them to \emph{study} too?

%% We have portrayed this criticism in somewhat facetious terms.  The
%% economic questions posed to CC by its critics are real, but building
%% programs that are not able to stand up for themselves and that are out
%% of touch with historical and modern trends in art (or other fields of
%% human endeavour) is hardly the answer.

The actual problem is that appreciation of computationally created
artefacts is \emph{hard}.  Consider the difference between creating a
video game (for example) and playing a video game.  In the first case,
the designer has full control over the rule-set, game mechanics,
interaction devices and so forth.  In the second case, we're more or
less in the world of general AI.  It is of course less untoward for a
computational video game designer to play its own games; this is state
of the art.

While the sketch of a solution developed here is by no means complete
even for poetry, we believe that extending capabilities from both
sides is robust.  However, as none of this has been tested in detailed
experiment, a fair criticism is that we do not know, yet, either how
much more work this approach will be, or how much better the results
will be.  We suspect it will be both harder, and worth it.  In the
following section we discuss the future outlook for the research approach.

%%% Local Variables: 
%%% mode: latex
%%% TeX-master: "mathsICCC"
%%% End: 

%% \section{Conclusions}
\label{sec:conc}

% BACKGROUND
% How has the social dimension of creativity been explored in CC to date? 
The ideas of social interaction, feedback, and evaluation have frequently been discussed in CC, but implementation
and theorisation around these topics has more limited.
% PHILOSOPHY AND METHODS
% How can a created artefact tell us about its making, and what can this contribute to CC?
In the current paper, we suggest giving artefacts more agency, designing computer programs with more autonomy, and focusing research on support for creative evolution.
% DESIGN
% How can computer poetry contribute to developing a process-based theory of poetics?
We have shown that in principle computers can engage in dialogue about poems, pointing to a theory of poetics rooted in the making of boundary-crossing objects and processes.
% FLOWR
% What would have to change about the FloWr system to implement the computational poetry workshop approach?
In order to move from thought experiment to computational simulation, FloWr could be helpfully extended with further programmer facilities including loops, subroutines, and commentaries, along with the ability to generate-and-test in a population-based manner, and the ability to learn over time.
% DISCUSSION
% What are the pros and cons of the workshop approach?
Workshops and related approaches are suitable for autonomous learning and development of the creative process, but they face technical and also some theoretical limitations.   
% CONCLUSIONS
% What might be the future role of dialogue in CC?
Dialogue may offer a way to creatively push these limits, empowering both programs and programmers.


%%% Local Variables: 
%%% mode: latex
%%% TeX-master: "mathsICCC"
%%% End: 


%% \section*{Acknowledgements} \label{sec:acknowledgements}
%% This research has been supported by EPSRC grants EP/L00206X and
%% EP/J004049, and the Future and Emerging Technologies (FET) programme
%% within the Seventh Framework Programme for Research of the European
%% Commission, under FET-Open Grant numbers: 611553 (COINVENT) and 611560
%% (WHIM).

\printbibliography

\end{document}
