\section{Introduction}
\label{sec:intro}

We should clarify at the outset that a central part of this paper is
not new.  We are writing in a large part to champion Alan Turing's
proposal that intelligent machines should ``be able to converse with
each other to sharpen their wits'' \cite{turing-intelligent}.  What is
new is the particular approach to computation that we advance in
support of this proposal.

While we do not precisely suggest an alternative to the
input/output/process model of computation \cite{marr1981vision}, we do
propose to extend it to build social effects.  A lot hinges on the
understanding of the term ``social.''  The formalism that is proposed
here builds on the notion of second order cybernetics that is
encapsulated by these two propositions of Heinz von Foerster's:

\begin{quote}
``Anything said is said \emph{by} an observer.'' \\
``Anything said is said \emph{to} an observer.''\\
\sourceatright{\cite{von2003cybernetics}}
\end{quote}

These propositions suggest a connection between observers, language,
and a relationship in which the language is used.  We call the
indicated relationship ``social.''  While the typical genre of design,
programming and \emph{first order cybernetics} is concerned with an
observer who specifies a system's purpose (and its inner workings),
the aim in \emph{second-order cybernetics} is for a system participant
to specify its own purpose in a ``social'' manner, that is, in
relationship to the rest of the system.

This is closely related to W. Brian Arthur's way of thinking about the
science of complexity.  He suggests that for sufficiently complex
problems, there is no correct statement of the problem.  ``All you can
say is that you have this situation and there are many ways to cognize
it'' \citep{brian-arthur-interview}.  The consequence is that:

\begin{quote}
Complexity is looking at interacting elements and asking how they form
patterns and how the patterns unfold.  It’s important to point out
that the patterns may never be finished.  They're open-ended.
\sourceatright{\citep{brian-arthur-interview}}
\end{quote}

The aim in this paper is to enhance the computer's response-ability,
its ability to identify its own approach to a given situation,
including its ability to identify and encode new patterns.  We can
potentially apply this sort thinking at each level of a program.  In
this way, the familiar model of input/output/process morphs to become
something more like \emph{context}/\emph{response}.  A response may be
in the form of ``output'' or it may be in the form of ``process'' or
it may be a mixture of these.  In particular, a satisfactory response
may be for the entity in question to change its own structure.
Another part of the response might change the context.  This could all
be viewed in a formal way as ``output'' -- new code may be written to
disk and update the entity's definition of ``self,'' for example --
but this isn't quite the same as the standard linear model that deals
with components that have been precisely defined in terms of its input
and output.  We still have components, but their behaviour can change
over time.

An approach like this will be necessary for building autonomous and
adaptive machines.  Equally, this context-driven approach seems
necessary for the development of a theory of \emph{social cybernetics}
as outlined by Von Foerster.  In the literature setting, similar
comments had been made over a decade earlier, by Mikhail Bakhtin, who
wrote:
\begin{quote}
The thinking human consciousness and the dialogic sphere in which this
consciousness exists, in all its depth and specificity, cannot be
reached through a monologic artistic approach.\\ \sourceatright{\citep[p. 271]{bakhtin1984problems}}
\end{quote}

Predicated on the assumption that \emph{computers are not human}
\citep[pp. 12, 18]{colton2012painting}, computers \emph{must} be
social, if they are to deal with problems of any great complexity
\citep{minsky1967programming,society-of-mind}.

This paper outlines an approach to social creativity, taking computer
poetry as our working domain.  It uses the Writer's Workshop model
\cite{gabriel2002writer} as its primary thought-architecture.  It will
be of interest to others working in the field of computational
creativity (hereafter, CC), who, we hope, will read it as an
invitation to a dialogue.

%%% Local Variables: 
%%% mode: latex
%%% TeX-master: "poetryICCC"
%%% End: 
