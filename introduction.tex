\section{Introduction}
\label{sec:intro}

% \begin{itemize}
% \item Why poetry?
% \item Why FloWr?
% \item Why Workshop?
% \end{itemize}

We are writing in a large part to champion Alan Turing's
proposal that intelligent machines should ``be able to converse with
each other to sharpen their wits'' \cite{turing-intelligent}. 
%
The formalism that we propose builds on the notion of social cybernetics that flows
from the following propositions from Heinz von Foerster, which he uses to theorise systems
in which participants can responsibly specify their own roles in relationship to
other system participants:

\begin{quote}
``Anything said is said \emph{by} an observer.'' \\
``Anything said is said \emph{to} an observer.''\\
\sourceatright{\cite{von2003cybernetics}}
\end{quote}

According to Jaako Seikkula and Tom Arnkil, who draw on the philosophical and literary analysis of Mikhail Bakhtin \cite{bakhtin2010toward,bakhtin1984problems} in their dialogical approach to psychosocial work,
\begin{quote}
``Dialogues could be called `the art of crossing boundaries'.  Instead of trying to control others, the parties reach out towards each other to hear their views better, to generate shared languages and to join resources.''
\sourceatright{\cite[p. 23]{seikkula2014open}}
\end{quote}

This paper outlines a study of social creativity with a dialogical emphasis, taking computer
poetry as our working domain.  It uses the Writer's Workshop model
\cite{gabriel2002writer} as the virtual laboratory in which to conduct a thought experiment.
The findings of our study are applied to the FloWr system \cite{charnley2014flowr}. 
We focus on the following questions in turn:
%\subsection{Outline}


\begin{itemize}[label=--,itemsep=0pt]
% BACKGROUND
% - a few more details
\item How has the social dimension of creativity been explored in CC to date? 
% PHILOSOPHY AND METHODS
% - break into subsections and add tools, integrate to table
%CG: Modified this to connect rosies methodology to the workshop and CC
\item How can a created artefact tell us about its making, and what can this contribute to CC?
% DESIGN
\item How can computer poetry contribute to developing a process-based theory of poetics? 
% FLOWR
% - Teresa to revise 
\item What would have to change about the FloWr system to implement the computational poetry workshop approach?
% DISCUSSION
% - keep it?
\item What are the pros and cons of the workshop approach?
% CONCLUSIONS
% - taking into account Christian's comment, I've revised this and will edit the conclusion to align.  JC.
\item What might be the future role of dialogue in CC?
\end{itemize}

%%% Not nec. needed
%%% In the background section, we will focus on social creativity in the computational creativity domain and more specifically in computer poetry. This will be followed by a Methods section, in which we ...

%%% Local Variables: 
%%% mode: latex
%%% TeX-master: "poetryICCC"
%%% End: 
